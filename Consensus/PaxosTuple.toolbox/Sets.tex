\batchmode %% Suppresses most terminal output.
\documentclass{article}
\usepackage{color}
\definecolor{boxshade}{gray}{0.85}
\setlength{\textwidth}{360pt}
\setlength{\textheight}{541pt}
\usepackage{latexsym}
\usepackage{ifthen}
% \usepackage{color}
%%%%%%%%%%%%%%%%%%%%%%%%%%%%%%%%%%%%%%%%%%%%%%%%%%%%%%%%%%%%%%%%%%%%%%%%%%%%%
% SWITCHES                                                                  %
%%%%%%%%%%%%%%%%%%%%%%%%%%%%%%%%%%%%%%%%%%%%%%%%%%%%%%%%%%%%%%%%%%%%%%%%%%%%%
\newboolean{shading} 
\setboolean{shading}{false}
\makeatletter
 %% this is needed only when inserted into the file, not when
 %% used as a package file.
%%%%%%%%%%%%%%%%%%%%%%%%%%%%%%%%%%%%%%%%%%%%%%%%%%%%%%%%%%%%%%%%%%%%%%%%%%%%%
%                                                                           %
% DEFINITIONS OF SYMBOL-PRODUCING COMMANDS                                  %
%                                                                           %
%    TLA+      LaTeX                                                        %
%    symbol    command                                                      %
%    ------    -------                                                      %
%    =>        \implies                                                     %
%    <:        \ltcolon                                                     %
%    :>        \colongt                                                     %
%    ==        \defeq                                                       %
%    ..        \dotdot                                                      %
%    ::        \coloncolon                                                  %
%    =|        \eqdash                                                      %
%    ++        \pp                                                          %
%    --        \mm                                                          %
%    **        \stst                                                        %
%    //        \slsl                                                        %
%    ^         \ct                                                          %
%    \A        \A                                                           %
%    \E        \E                                                           %
%    \AA       \AA                                                          %
%    \EE       \EE                                                          %
%%%%%%%%%%%%%%%%%%%%%%%%%%%%%%%%%%%%%%%%%%%%%%%%%%%%%%%%%%%%%%%%%%%%%%%%%%%%%
\newlength{\symlength}
\newcommand{\implies}{\Rightarrow}
\newcommand{\ltcolon}{\mathrel{<\!\!\mbox{:}}}
\newcommand{\colongt}{\mathrel{\!\mbox{:}\!\!>}}
\newcommand{\defeq}{\;\mathrel{\smash   %% keep this symbol from being too tall
    {{\stackrel{\scriptscriptstyle\Delta}{=}}}}\;}
\newcommand{\dotdot}{\mathrel{\ldotp\ldotp}}
\newcommand{\coloncolon}{\mathrel{::\;}}
\newcommand{\eqdash}{\mathrel = \joinrel \hspace{-.28em}|}
\newcommand{\pp}{\mathbin{++}}
\newcommand{\mm}{\mathbin{--}}
\newcommand{\stst}{*\!*}
\newcommand{\slsl}{/\!/}
\newcommand{\ct}{\hat{\hspace{.4em}}}
\newcommand{\A}{\forall}
\newcommand{\E}{\exists}
\renewcommand{\AA}{\makebox{$\raisebox{.05em}{\makebox[0pt][l]{%
   $\forall\hspace{-.517em}\forall\hspace{-.517em}\forall$}}%
   \forall\hspace{-.517em}\forall \hspace{-.517em}\forall\,$}}
\newcommand{\EE}{\makebox{$\raisebox{.05em}{\makebox[0pt][l]{%
   $\exists\hspace{-.517em}\exists\hspace{-.517em}\exists$}}%
   \exists\hspace{-.517em}\exists\hspace{-.517em}\exists\,$}}
\newcommand{\whileop}{\.{\stackrel
  {\mbox{\raisebox{-.3em}[0pt][0pt]{$\scriptscriptstyle+\;\,$}}}%
  {-\hspace{-.16em}\triangleright}}}

% Commands are defined to produce the upper-case keywords.
% Note that some have space after them.
\newcommand{\ASSUME}{\textsc{assume }}
\newcommand{\ASSUMPTION}{\textsc{assumption }}
\newcommand{\AXIOM}{\textsc{axiom }}
\newcommand{\BOOLEAN}{\textsc{boolean }}
\newcommand{\CASE}{\textsc{case }}
\newcommand{\CONSTANT}{\textsc{constant }}
\newcommand{\CONSTANTS}{\textsc{constants }}
\newcommand{\ELSE}{\settowidth{\symlength}{\THEN}%
   \makebox[\symlength][l]{\textsc{ else}}}
\newcommand{\EXCEPT}{\textsc{ except }}
\newcommand{\EXTENDS}{\textsc{extends }}
\newcommand{\FALSE}{\textsc{false}}
\newcommand{\IF}{\textsc{if }}
\newcommand{\IN}{\settowidth{\symlength}{\LET}%
   \makebox[\symlength][l]{\textsc{in}}}
\newcommand{\INSTANCE}{\textsc{instance }}
\newcommand{\LET}{\textsc{let }}
\newcommand{\LOCAL}{\textsc{local }}
\newcommand{\MODULE}{\textsc{module }}
\newcommand{\OTHER}{\textsc{other }}
\newcommand{\STRING}{\textsc{string}}
\newcommand{\THEN}{\textsc{ then }}
\newcommand{\THEOREM}{\textsc{theorem }}
\newcommand{\LEMMA}{\textsc{lemma }}
\newcommand{\PROPOSITION}{\textsc{proposition }}
\newcommand{\COROLLARY}{\textsc{corollary }}
\newcommand{\TRUE}{\textsc{true}}
\newcommand{\VARIABLE}{\textsc{variable }}
\newcommand{\VARIABLES}{\textsc{variables }}
\newcommand{\WITH}{\textsc{ with }}
\newcommand{\WF}{\textrm{WF}}
\newcommand{\SF}{\textrm{SF}}
\newcommand{\CHOOSE}{\textsc{choose }}
\newcommand{\ENABLED}{\textsc{enabled }}
\newcommand{\UNCHANGED}{\textsc{unchanged }}
\newcommand{\SUBSET}{\textsc{subset }}
\newcommand{\UNION}{\textsc{union }}
\newcommand{\DOMAIN}{\textsc{domain }}
% Added for tla2tex
\newcommand{\BY}{\textsc{by }}
\newcommand{\OBVIOUS}{\textsc{obvious }}
\newcommand{\HAVE}{\textsc{have }}
\newcommand{\QED}{\textsc{qed }}
\newcommand{\TAKE}{\textsc{take }}
\newcommand{\DEF}{\textsc{ def }}
\newcommand{\HIDE}{\textsc{hide }}
\newcommand{\RECURSIVE}{\textsc{recursive }}
\newcommand{\USE}{\textsc{use }}
\newcommand{\DEFINE}{\textsc{define }}
\newcommand{\PROOF}{\textsc{proof }}
\newcommand{\WITNESS}{\textsc{witness }}
\newcommand{\PICK}{\textsc{pick }}
\newcommand{\DEFS}{\textsc{defs }}
\newcommand{\PROVE}{\settowidth{\symlength}{\ASSUME}%
   \makebox[\symlength][l]{\textsc{prove}}\@s{-4.1}}%
  %% The \@s{-4.1) is a kludge added on 24 Oct 2009 [happy birthday, Ellen]
  %% so the correct alignment occurs if the user types
  %%   ASSUME X
  %%   PROVE  Y
  %% because it cancels the extra 4.1 pts added because of the 
  %% extra space after the PROVE.  This seems to works OK.
  %% However, the 4.1 equals Parameters.LaTeXLeftSpace(1) and
  %% should be changed if that method ever changes.
\newcommand{\SUFFICES}{\textsc{suffices }}
\newcommand{\NEW}{\textsc{new }}
\newcommand{\LAMBDA}{\textsc{lambda }}
\newcommand{\STATE}{\textsc{state }}
\newcommand{\ACTION}{\textsc{action }}
\newcommand{\TEMPORAL}{\textsc{temporal }}
\newcommand{\ONLY}{\textsc{only }}              %% added by LL on 2 Oct 2009
\newcommand{\OMITTED}{\textsc{omitted }}        %% added by LL on 31 Oct 2009
\newcommand{\@pfstepnum}[2]{\ensuremath{\langle#1\rangle}\textrm{#2}}
\newcommand{\bang}{\@s{1}\mbox{\small !}\@s{1}}
%% We should format || differently in PlusCal code than in TLA+ formulas.
\newcommand{\p@barbar}{\ifpcalsymbols
   \,\,\rule[-.25em]{.075em}{1em}\hspace*{.2em}\rule[-.25em]{.075em}{1em}\,\,%
   \else \,||\,\fi}
%% PlusCal keywords
\newcommand{\p@fair}{\textbf{fair }}
\newcommand{\p@semicolon}{\textbf{\,; }}
\newcommand{\p@algorithm}{\textbf{algorithm }}
\newcommand{\p@mmfair}{\textbf{-{}-fair }}
\newcommand{\p@mmalgorithm}{\textbf{-{}-algorithm }}
\newcommand{\p@assert}{\textbf{assert }}
\newcommand{\p@await}{\textbf{await }}
\newcommand{\p@begin}{\textbf{begin }}
\newcommand{\p@end}{\textbf{end }}
\newcommand{\p@call}{\textbf{call }}
\newcommand{\p@define}{\textbf{define }}
\newcommand{\p@do}{\textbf{ do }}
\newcommand{\p@either}{\textbf{either }}
\newcommand{\p@or}{\textbf{or }}
\newcommand{\p@goto}{\textbf{goto }}
\newcommand{\p@if}{\textbf{if }}
\newcommand{\p@then}{\,\,\textbf{then }}
\newcommand{\p@else}{\ifcsyntax \textbf{else } \else \,\,\textbf{else }\fi}
\newcommand{\p@elsif}{\,\,\textbf{elsif }}
\newcommand{\p@macro}{\textbf{macro }}
\newcommand{\p@print}{\textbf{print }}
\newcommand{\p@procedure}{\textbf{procedure }}
\newcommand{\p@process}{\textbf{process }}
\newcommand{\p@return}{\textbf{return}}
\newcommand{\p@skip}{\textbf{skip}}
\newcommand{\p@variable}{\textbf{variable }}
\newcommand{\p@variables}{\textbf{variables }}
\newcommand{\p@while}{\textbf{while }}
\newcommand{\p@when}{\textbf{when }}
\newcommand{\p@with}{\textbf{with }}
\newcommand{\p@lparen}{\textbf{(\,\,}}
\newcommand{\p@rparen}{\textbf{\,\,) }}   
\newcommand{\p@lbrace}{\textbf{\{\,\,}}   
\newcommand{\p@rbrace}{\textbf{\,\,\} }}

%%%%%%%%%%%%%%%%%%%%%%%%%%%%%%%%%%%%%%%%%%%%%%%%%%%%%%%%%
% REDEFINE STANDARD COMMANDS TO MAKE THEM FORMAT BETTER %
%                                                       %
% We redefine \in and \notin                            %
%%%%%%%%%%%%%%%%%%%%%%%%%%%%%%%%%%%%%%%%%%%%%%%%%%%%%%%%%
\renewcommand{\_}{\rule{.4em}{.06em}\hspace{.05em}}
\newlength{\equalswidth}
\let\oldin=\in
\let\oldnotin=\notin
\renewcommand{\in}{%
   {\settowidth{\equalswidth}{$\.{=}$}\makebox[\equalswidth][c]{$\oldin$}}}
\renewcommand{\notin}{%
   {\settowidth{\equalswidth}{$\.{=}$}\makebox[\equalswidth]{$\oldnotin$}}}


%%%%%%%%%%%%%%%%%%%%%%%%%%%%%%%%%%%%%%%%%%%%%%%%%%%%
%                                                  %
% HORIZONTAL BARS:                                 %
%                                                  %
%   \moduleLeftDash    |~~~~~~~~~~                 %
%   \moduleRightDash    ~~~~~~~~~~|                %
%   \midbar            |----------|                %
%   \bottombar         |__________|                %
%%%%%%%%%%%%%%%%%%%%%%%%%%%%%%%%%%%%%%%%%%%%%%%%%%%%
\newlength{\charwidth}\settowidth{\charwidth}{{\small\tt M}}
\newlength{\boxrulewd}\setlength{\boxrulewd}{.4pt}
\newlength{\boxlineht}\setlength{\boxlineht}{.5\baselineskip}
\newcommand{\boxsep}{\charwidth}
\newlength{\boxruleht}\setlength{\boxruleht}{.5ex}
\newlength{\boxruledp}\setlength{\boxruledp}{-\boxruleht}
\addtolength{\boxruledp}{\boxrulewd}
\newcommand{\boxrule}{\leaders\hrule height \boxruleht depth \boxruledp
                      \hfill\mbox{}}
\newcommand{\@computerule}{%
  \setlength{\boxruleht}{.5ex}%
  \setlength{\boxruledp}{-\boxruleht}%
  \addtolength{\boxruledp}{\boxrulewd}}

\newcommand{\bottombar}{\hspace{-\boxsep}%
  \raisebox{-\boxrulewd}[0pt][0pt]{\rule[.5ex]{\boxrulewd}{\boxlineht}}%
  \boxrule
  \raisebox{-\boxrulewd}[0pt][0pt]{%
      \rule[.5ex]{\boxrulewd}{\boxlineht}}\hspace{-\boxsep}\vspace{0pt}}

\newcommand{\moduleLeftDash}%
   {\hspace*{-\boxsep}%
     \raisebox{-\boxlineht}[0pt][0pt]{\rule[.5ex]{\boxrulewd
               }{\boxlineht}}%
    \boxrule\hspace*{.4em }}

\newcommand{\moduleRightDash}%
    {\hspace*{.4em}\boxrule
    \raisebox{-\boxlineht}[0pt][0pt]{\rule[.5ex]{\boxrulewd
               }{\boxlineht}}\hspace{-\boxsep}}%\vspace{.2em}

\newcommand{\midbar}{\hspace{-\boxsep}\raisebox{-.5\boxlineht}[0pt][0pt]{%
   \rule[.5ex]{\boxrulewd}{\boxlineht}}\boxrule\raisebox{-.5\boxlineht%
   }[0pt][0pt]{\rule[.5ex]{\boxrulewd}{\boxlineht}}\hspace{-\boxsep}}

%%%%%%%%%%%%%%%%%%%%%%%%%%%%%%%%%%%%%%%%%%%%%%%%%%%%%%%%%%%%%%%%%%%%%%%%%%%%%
% FORMATING COMMANDS                                                        %
%%%%%%%%%%%%%%%%%%%%%%%%%%%%%%%%%%%%%%%%%%%%%%%%%%%%%%%%%%%%%%%%%%%%%%%%%%%%%

%%%%%%%%%%%%%%%%%%%%%%%%%%%%%%%%%%%%%%%%%%%%%%%%%%%%%%%%%%%%%%%%%%%%%%%%%%%%%
% PLUSCAL SHADING                                                           %
%%%%%%%%%%%%%%%%%%%%%%%%%%%%%%%%%%%%%%%%%%%%%%%%%%%%%%%%%%%%%%%%%%%%%%%%%%%%%

% The TeX pcalshading switch is set on to cause PlusCal shading to be
% performed.  This changes the behavior of the following commands and
% environments to cause full-width shading to be performed on all lines.
% 
%   \tstrut \@x cpar mcom \@pvspace
% 
% The TeX pcalsymbols switch is turned on when typesetting a PlusCal algorithm,
% whether or not shading is being performed.  It causes symbols (other than
% parentheses and braces and PlusCal-only keywords) that should be typeset
% differently depending on whether they are in an algorithm to be typeset
% appropriately.  Currently, the only such symbol is "||".
%
% The TeX csyntax switch is turned on when typesetting a PlusCal algorithm in
% c-syntax.  This allows symbols to be format differently in the two syntaxes.
% The "else" keyword is the only one that is.

\newif\ifpcalshading \pcalshadingfalse
\newif\ifpcalsymbols \pcalsymbolsfalse
\newif\ifcsyntax     \csyntaxtrue

% The \@pvspace command makes a vertical space.  It uses \vspace
% except with \ifpcalshading, in which case it sets \pvcalvspace
% and the space is added by a following \@x command.
%
\newlength{\pcalvspace}\setlength{\pcalvspace}{0pt}%
\newcommand{\@pvspace}[1]{%
  \ifpcalshading
     \par\global\setlength{\pcalvspace}{#1}%
  \else
     \par\vspace{#1}%
  \fi
}

% The lcom environment was changed to set \lcomindent equal to
% the indentation it produces.  This length is used by the
% cpar environment to make shading extend for the full width
% of the line.  This assumes that lcom environments are not
% nested.  I hope TLATeX does not nest them.
%
\newlength{\lcomindent}%
\setlength{\lcomindent}{0pt}%

%\tstrut: A strut to produce inter-paragraph space in a comment.
%\rstrut: A strut to extend the bottom of a one-line comment so
%         there's no break in the shading between comments on 
%         successive lines.
\newcommand\tstrut%
  {\raisebox{\vshadelen}{\raisebox{-.25em}{\rule{0pt}{1.15em}}}%
   \global\setlength{\vshadelen}{0pt}}
\newcommand\rstrut{\raisebox{-.25em}{\rule{0pt}{1.15em}}%
 \global\setlength{\vshadelen}{0pt}}


% \.{op} formats operator op in math mode with empty boxes on either side.
% Used because TeX otherwise vary the amount of space it leaves around op.
\renewcommand{\.}[1]{\ensuremath{\mbox{}#1\mbox{}}}

% \@s{n} produces an n-point space
\newcommand{\@s}[1]{\hspace{#1pt}}           

% \@x{txt} starts a specification line in the beginning with txt
% in the final LaTeX source.
\newlength{\@xlen}
\newcommand\xtstrut%
  {\setlength{\@xlen}{1.05em}%
   \addtolength{\@xlen}{\pcalvspace}%
    \raisebox{\vshadelen}{\raisebox{-.25em}{\rule{0pt}{\@xlen}}}%
   \global\setlength{\vshadelen}{0pt}%
   \global\setlength{\pcalvspace}{0pt}}

\newcommand{\@x}[1]{\par
  \ifpcalshading
  \makebox[0pt][l]{\shadebox{\xtstrut\hspace*{\textwidth}}}%
  \fi
  \mbox{$\mbox{}#1\mbox{}$}}  

% \@xx{txt} continues a specification line with the text txt.
\newcommand{\@xx}[1]{\mbox{$\mbox{}#1\mbox{}$}}  

% \@y{cmt} produces a one-line comment.
\newcommand{\@y}[1]{\mbox{\footnotesize\hspace{.65em}%
  \ifthenelse{\boolean{shading}}{%
      \shadebox{#1\hspace{-\the\lastskip}\rstrut}}%
               {#1\hspace{-\the\lastskip}\rstrut}}}

% \@z{cmt} produces a zero-width one-line comment.
\newcommand{\@z}[1]{\makebox[0pt][l]{\footnotesize
  \ifthenelse{\boolean{shading}}{%
      \shadebox{#1\hspace{-\the\lastskip}\rstrut}}%
               {#1\hspace{-\the\lastskip}\rstrut}}}


% \@w{str} produces the TLA+ string "str".
\newcommand{\@w}[1]{\textsf{``{#1}''}}             


%%%%%%%%%%%%%%%%%%%%%%%%%%%%%%%%%%%%%%%%%%%%%%%%%%%%%%%%%%%%%%%%%%%%%%%%%%%%%
% SHADING                                                                   %
%%%%%%%%%%%%%%%%%%%%%%%%%%%%%%%%%%%%%%%%%%%%%%%%%%%%%%%%%%%%%%%%%%%%%%%%%%%%%
\def\graymargin{1}
  % The number of points of margin in the shaded box.

% \definecolor{boxshade}{gray}{.85}
% Defines the darkness of the shading: 1 = white, 0 = black
% Added by TLATeX only if needed.

% \shadebox{txt} puts txt in a shaded box.
\newlength{\templena}
\newlength{\templenb}
\newsavebox{\tempboxa}
\newcommand{\shadebox}[1]{{\setlength{\fboxsep}{\graymargin pt}%
     \savebox{\tempboxa}{#1}%
     \settoheight{\templena}{\usebox{\tempboxa}}%
     \settodepth{\templenb}{\usebox{\tempboxa}}%
     \hspace*{-\fboxsep}\raisebox{0pt}[\templena][\templenb]%
        {\colorbox{boxshade}{\usebox{\tempboxa}}}\hspace*{-\fboxsep}}}

% \vshade{n} makes an n-point inter-paragraph space, with
%  shading if the `shading' flag is true.
\newlength{\vshadelen}
\setlength{\vshadelen}{0pt}
\newcommand{\vshade}[1]{\ifthenelse{\boolean{shading}}%
   {\global\setlength{\vshadelen}{#1pt}}%
   {\vspace{#1pt}}}

\newlength{\boxwidth}
\newlength{\multicommentdepth}

%%%%%%%%%%%%%%%%%%%%%%%%%%%%%%%%%%%%%%%%%%%%%%%%%%%%%%%%%%%%%%%%%%%%%%%%%%%%%
% THE cpar ENVIRONMENT                                                      %
% ^^^^^^^^^^^^^^^^^^^^                                                      %
% The LaTeX input                                                           %
%                                                                           %
%   \begin{cpar}{pop}{nest}{isLabel}{d}{e}{arg6}                            %
%     XXXXXXXXXXXXXXX                                                       %
%     XXXXXXXXXXXXXXX                                                       %
%     XXXXXXXXXXXXXXX                                                       %
%   \end{cpar}                                                              %
%                                                                           %
% produces one of two possible results.  If isLabel is the letter "T",      %
% it produces the following, where [label] is the result of typesetting     %
% arg6 in an LR box, and d is is a number representing a distance in        %
% points.                                                                   %
%                                                                           %
%   prevailing |<-- d -->[label]<- e ->XXXXXXXXXXXXXXX                      %
%         left |                       XXXXXXXXXXXXXXX                      %
%       margin |                       XXXXXXXXXXXXXXX                      %
%                                                                           %
% If isLabel is the letter "F", then it produces                            %
%                                                                           %
%   prevailing |<-- d -->XXXXXXXXXXXXXXXXXXXXXXX                            %
%         left |         <- e ->XXXXXXXXXXXXXXXX                            %
%       margin |                XXXXXXXXXXXXXXXX                            %
%                                                                           %
% where d and e are numbers representing distances in points.               %
%                                                                           %
% The prevailing left margin is the one in effect before the most recent    %
% pop (argument 1) cpar environments with "T" as the nest argument, where   %
% pop is a number \geq 0.                                                   %
%                                                                           %
% If the nest argument is the letter "T", then the prevailing left          %
% margin is moved to the left of the second (and following) lines of        %
% X's.  Otherwise, the prevailing left margin is left unchanged.            %
%                                                                           %
% An \unnest{n} command moves the prevailing left margin to where it was    %
% before the most recent n cpar environments with "T" as the nesting        %
% argument.                                                                 %
%                                                                           %
% The environment leaves no vertical space above or below it, or between    %
% its paragraphs.  (TLATeX inserts the proper amount of vertical space.)    %
%%%%%%%%%%%%%%%%%%%%%%%%%%%%%%%%%%%%%%%%%%%%%%%%%%%%%%%%%%%%%%%%%%%%%%%%%%%%%

\newcounter{pardepth}
\setcounter{pardepth}{0}

% \setgmargin{txt} defines \gmarginN to be txt, where N is \roman{pardepth}.
% \thegmargin equals \gmarginN, where N is \roman{pardepth}.
\newcommand{\setgmargin}[1]{%
  \expandafter\xdef\csname gmargin\roman{pardepth}\endcsname{#1}}
\newcommand{\thegmargin}{\csname gmargin\roman{pardepth}\endcsname}
\newcommand{\gmargin}{0pt}

\newsavebox{\tempsbox}

\newlength{\@cparht}
\newlength{\@cpardp}
\newenvironment{cpar}[6]{%
  \addtocounter{pardepth}{-#1}%
  \ifthenelse{\boolean{shading}}{\par\begin{lrbox}{\tempsbox}%
                                 \begin{minipage}[t]{\linewidth}}{}%
  \begin{list}{}{%
     \edef\temp{\thegmargin}
     \ifthenelse{\equal{#3}{T}}%
       {\settowidth{\leftmargin}{\hspace{\temp}\footnotesize #6\hspace{#5pt}}%
        \addtolength{\leftmargin}{#4pt}}%
       {\setlength{\leftmargin}{#4pt}%
        \addtolength{\leftmargin}{#5pt}%
        \addtolength{\leftmargin}{\temp}%
        \setlength{\itemindent}{-#5pt}}%
      \ifthenelse{\equal{#2}{T}}{\addtocounter{pardepth}{1}%
                                 \setgmargin{\the\leftmargin}}{}%
      \setlength{\labelwidth}{0pt}%
      \setlength{\labelsep}{0pt}%
      \setlength{\itemindent}{-\leftmargin}%
      \setlength{\topsep}{0pt}%
      \setlength{\parsep}{0pt}%
      \setlength{\partopsep}{0pt}%
      \setlength{\parskip}{0pt}%
      \setlength{\itemsep}{0pt}
      \setlength{\itemindent}{#4pt}%
      \addtolength{\itemindent}{-\leftmargin}}%
   \ifthenelse{\equal{#3}{T}}%
      {\item[\tstrut\footnotesize \hspace{\temp}{#6}\hspace{#5pt}]
        }%
      {\item[\tstrut\hspace{\temp}]%
         }%
   \footnotesize}
 {\hspace{-\the\lastskip}\tstrut
 \end{list}%
  \ifthenelse{\boolean{shading}}%
          {\end{minipage}%
           \end{lrbox}%
           \ifpcalshading
             \setlength{\@cparht}{\ht\tempsbox}%
             \setlength{\@cpardp}{\dp\tempsbox}%
             \addtolength{\@cparht}{.15em}%
             \addtolength{\@cpardp}{.2em}%
             \addtolength{\@cparht}{\@cpardp}%
            % I don't know what's going on here.  I want to add a
            % \pcalvspace high shaded line, but I don't know how to
            % do it.  A little trial and error shows that the following
            % does a reasonable job approximating that, eliminating
            % the line if \pcalvspace is small.
            \addtolength{\@cparht}{\pcalvspace}%
             \ifdim \pcalvspace > .8em
               \addtolength{\pcalvspace}{-.2em}%
               \hspace*{-\lcomindent}%
               \shadebox{\rule{0pt}{\pcalvspace}\hspace*{\textwidth}}\par
               \global\setlength{\pcalvspace}{0pt}%
               \fi
             \hspace*{-\lcomindent}%
             \makebox[0pt][l]{\raisebox{-\@cpardp}[0pt][0pt]{%
                 \shadebox{\rule{0pt}{\@cparht}\hspace*{\textwidth}}}}%
             \hspace*{\lcomindent}\usebox{\tempsbox}%
             \par
           \else
             \shadebox{\usebox{\tempsbox}}\par
           \fi}%
           {}%
  }

%%%%%%%%%%%%%%%%%%%%%%%%%%%%%%%%%%%%%%%%%%%%%%%%%%%%%%%%%%%%%%%%%%%%%%%%%%%%%%
% THE ppar ENVIRONMENT                                                       %
% ^^^^^^^^^^^^^^^^^^^^                                                       %
% The environment                                                            %
%                                                                            %
%   \begin{ppar} ... \end{ppar}                                              %
%                                                                            %
% is equivalent to                                                           %
%                                                                            %
%   \begin{cpar}{0}{F}{F}{0}{0}{} ... \end{cpar}                             %
%                                                                            %
% The environment is put around each line of the output for a PlusCal        %
% algorithm.                                                                 %
%%%%%%%%%%%%%%%%%%%%%%%%%%%%%%%%%%%%%%%%%%%%%%%%%%%%%%%%%%%%%%%%%%%%%%%%%%%%%%
%\newenvironment{ppar}{%
%  \ifthenelse{\boolean{shading}}{\par\begin{lrbox}{\tempsbox}%
%                                 \begin{minipage}[t]{\linewidth}}{}%
%  \begin{list}{}{%
%     \edef\temp{\thegmargin}
%        \setlength{\leftmargin}{0pt}%
%        \addtolength{\leftmargin}{\temp}%
%        \setlength{\itemindent}{0pt}%
%      \setlength{\labelwidth}{0pt}%
%      \setlength{\labelsep}{0pt}%
%      \setlength{\itemindent}{-\leftmargin}%
%      \setlength{\topsep}{0pt}%
%      \setlength{\parsep}{0pt}%
%      \setlength{\partopsep}{0pt}%
%      \setlength{\parskip}{0pt}%
%      \setlength{\itemsep}{0pt}
%      \setlength{\itemindent}{0pt}%
%      \addtolength{\itemindent}{-\leftmargin}}%
%      \item[\tstrut\hspace{\temp}]}%
% {\hspace{-\the\lastskip}\tstrut
% \end{list}%
%  \ifthenelse{\boolean{shading}}{\end{minipage}  
%                                 \end{lrbox}%
%                                 \shadebox{\usebox{\tempsbox}}\par}{}%
%  }

 %%% TESTING
 \newcommand{\xtest}[1]{\par
 \makebox[0pt][l]{\shadebox{\xtstrut\hspace*{\textwidth}}}%
 \mbox{$\mbox{}#1\mbox{}$}} 

% \newcommand{\xxtest}[1]{\par
% \makebox[0pt][l]{\shadebox{\xtstrut{#1}\hspace*{\textwidth}}}%
% \mbox{$\mbox{}#1\mbox{}$}} 

%\newlength{\pcalvspace}
%\setlength{\pcalvspace}{0pt}
% \newlength{\xxtestlen}
% \setlength{\xxtestlen}{0pt}
% \newcommand\xtstrut%
%   {\setlength{\xxtestlen}{1.15em}%
%    \addtolength{\xxtestlen}{\pcalvspace}%
%     \raisebox{\vshadelen}{\raisebox{-.25em}{\rule{0pt}{\xxtestlen}}}%
%    \global\setlength{\vshadelen}{0pt}%
%    \global\setlength{\pcalvspace}{0pt}}
   
   %%%% TESTING
   
   %% The xcpar environment
   %%  Note: overloaded use of \pcalvspace for testing.
   %%
%   \newlength{\xcparht}%
%   \newlength{\xcpardp}%
   
%   \newenvironment{xcpar}[6]{%
%  \addtocounter{pardepth}{-#1}%
%  \ifthenelse{\boolean{shading}}{\par\begin{lrbox}{\tempsbox}%
%                                 \begin{minipage}[t]{\linewidth}}{}%
%  \begin{list}{}{%
%     \edef\temp{\thegmargin}%
%     \ifthenelse{\equal{#3}{T}}%
%       {\settowidth{\leftmargin}{\hspace{\temp}\footnotesize #6\hspace{#5pt}}%
%        \addtolength{\leftmargin}{#4pt}}%
%       {\setlength{\leftmargin}{#4pt}%
%        \addtolength{\leftmargin}{#5pt}%
%        \addtolength{\leftmargin}{\temp}%
%        \setlength{\itemindent}{-#5pt}}%
%      \ifthenelse{\equal{#2}{T}}{\addtocounter{pardepth}{1}%
%                                 \setgmargin{\the\leftmargin}}{}%
%      \setlength{\labelwidth}{0pt}%
%      \setlength{\labelsep}{0pt}%
%      \setlength{\itemindent}{-\leftmargin}%
%      \setlength{\topsep}{0pt}%
%      \setlength{\parsep}{0pt}%
%      \setlength{\partopsep}{0pt}%
%      \setlength{\parskip}{0pt}%
%      \setlength{\itemsep}{0pt}%
%      \setlength{\itemindent}{#4pt}%
%      \addtolength{\itemindent}{-\leftmargin}}%
%   \ifthenelse{\equal{#3}{T}}%
%      {\item[\xtstrut\footnotesize \hspace{\temp}{#6}\hspace{#5pt}]%
%        }%
%      {\item[\xtstrut\hspace{\temp}]%
%         }%
%   \footnotesize}
% {\hspace{-\the\lastskip}\tstrut
% \end{list}%
%  \ifthenelse{\boolean{shading}}{\end{minipage}  
%                                 \end{lrbox}%
%                                 \setlength{\xcparht}{\ht\tempsbox}%
%                                 \setlength{\xcpardp}{\dp\tempsbox}%
%                                 \addtolength{\xcparht}{.15em}%
%                                 \addtolength{\xcpardp}{.2em}%
%                                 \addtolength{\xcparht}{\xcpardp}%
%                                 \hspace*{-\lcomindent}%
%                                 \makebox[0pt][l]{\raisebox{-\xcpardp}[0pt][0pt]{%
%                                      \shadebox{\rule{0pt}{\xcparht}\hspace*{\textwidth}}}}%
%                                 \hspace*{\lcomindent}\usebox{\tempsbox}%
%                                 \par}{}%
%  }
%  
% \newlength{\xmcomlen}
%\newenvironment{xmcom}[1]{%
%  \setcounter{pardepth}{0}%
%  \hspace{.65em}%
%  \begin{lrbox}{\alignbox}\sloppypar%
%      \setboolean{shading}{false}%
%      \setlength{\boxwidth}{#1pt}%
%      \addtolength{\boxwidth}{-.65em}%
%      \begin{minipage}[t]{\boxwidth}\footnotesize
%      \parskip=0pt\relax}%
%       {\end{minipage}\end{lrbox}%
%       \setlength{\xmcomlen}{\textwidth}%
%       \addtolength{\xmcomlen}{-\wd\alignbox}%
%       \settodepth{\alignwidth}{\usebox{\alignbox}}%
%       \global\setlength{\multicommentdepth}{\alignwidth}%
%       \setlength{\boxwidth}{\alignwidth}%
%       \global\addtolength{\alignwidth}{-\maxdepth}%
%       \addtolength{\boxwidth}{.1em}%
%       \raisebox{0pt}[0pt][0pt]{%
%        \ifthenelse{\boolean{shading}}%
%          {\hspace*{-\xmcomlen}\shadebox{\rule[-\boxwidth]{0pt}{0pt}%
%                                 \hspace*{\xmcomlen}\usebox{\alignbox}}}%
%          {\usebox{\alignbox}}}%
%       \vspace*{\alignwidth}\pagebreak[0]\vspace{-\alignwidth}\par}
% % a multi-line comment, whose first argument is its width in points.
%  
   
%%%%%%%%%%%%%%%%%%%%%%%%%%%%%%%%%%%%%%%%%%%%%%%%%%%%%%%%%%%%%%%%%%%%%%%%%%%%%%
% THE lcom ENVIRONMENT                                                       %
% ^^^^^^^^^^^^^^^^^^^^                                                       %
% A multi-line comment with no text to its left is typeset in an lcom        % 
% environment, whose argument is a number representing the indentation       % 
% of the left margin, in points.  All the text of the comment should be      % 
% inside cpar environments.                                                  % 
%%%%%%%%%%%%%%%%%%%%%%%%%%%%%%%%%%%%%%%%%%%%%%%%%%%%%%%%%%%%%%%%%%%%%%%%%%%%%%
\newenvironment{lcom}[1]{%
  \setlength{\lcomindent}{#1pt} % Added for PlusCal handling.
  \par\vspace{.2em}%
  \sloppypar
  \setcounter{pardepth}{0}%
  \footnotesize
  \begin{list}{}{%
    \setlength{\leftmargin}{#1pt}
    \setlength{\labelwidth}{0pt}%
    \setlength{\labelsep}{0pt}%
    \setlength{\itemindent}{0pt}%
    \setlength{\topsep}{0pt}%
    \setlength{\parsep}{0pt}%
    \setlength{\partopsep}{0pt}%
    \setlength{\parskip}{0pt}}
    \item[]}%
  {\end{list}\vspace{.3em}\setlength{\lcomindent}{0pt}%
 }


%%%%%%%%%%%%%%%%%%%%%%%%%%%%%%%%%%%%%%%%%%%%%%%%%%%%%%%%%%%%%%%%%%%%%%%%%%%%%
% THE mcom ENVIRONMENT AND \mutivspace COMMAND                              %
% ^^^^^^^^^^^^^^^^^^^^^^^^^^^^^^^^^^^^^^^^^^^^                              %
%                                                                           %
% A part of the spec containing a right-comment of the form                 %
%                                                                           %
%      xxxx (*************)                                                 %
%      yyyy (* ccccccccc *)                                                 %
%      ...  (* ccccccccc *)                                                 %
%           (* ccccccccc *)                                                 %
%           (* ccccccccc *)                                                 %
%           (*************)                                                 %
%                                                                           %
% is typeset by                                                             %
%                                                                           %
%     XXXX \begin{mcom}{d}                                                  %
%            CCCC ... CCC                                                   %
%          \end{mcom}                                                       %
%     YYYY ...                                                              %
%     \multivspace{n}                                                       %
%                                                                           %
% where the number d is the width in points of the comment, n is the        %
% number of xxxx, yyyy, ...  lines to the left of the comment.              %
% All the text of the comment should be typeset in cpar environments.       %
%                                                                           %
% This puts the comment into a single box (so no page breaks can occur      %
% within it).  The entire box is shaded iff the shading flag is true.       %
%%%%%%%%%%%%%%%%%%%%%%%%%%%%%%%%%%%%%%%%%%%%%%%%%%%%%%%%%%%%%%%%%%%%%%%%%%%%%
\newlength{\xmcomlen}%
\newenvironment{mcom}[1]{%
  \setcounter{pardepth}{0}%
  \hspace{.65em}%
  \begin{lrbox}{\alignbox}\sloppypar%
      \setboolean{shading}{false}%
      \setlength{\boxwidth}{#1pt}%
      \addtolength{\boxwidth}{-.65em}%
      \begin{minipage}[t]{\boxwidth}\footnotesize
      \parskip=0pt\relax}%
       {\end{minipage}\end{lrbox}%
       \setlength{\xmcomlen}{\textwidth}%       % For PlusCal shading
       \addtolength{\xmcomlen}{-\wd\alignbox}%  % For PlusCal shading
       \settodepth{\alignwidth}{\usebox{\alignbox}}%
       \global\setlength{\multicommentdepth}{\alignwidth}%
       \setlength{\boxwidth}{\alignwidth}%      % For PlusCal shading
       \global\addtolength{\alignwidth}{-\maxdepth}%
       \addtolength{\boxwidth}{.1em}%           % For PlusCal shading
      \raisebox{0pt}[0pt][0pt]{%
        \ifthenelse{\boolean{shading}}%
          {\ifpcalshading
             \hspace*{-\xmcomlen}%
             \shadebox{\rule[-\boxwidth]{0pt}{0pt}\hspace*{\xmcomlen}%
                          \usebox{\alignbox}}%
           \else
             \shadebox{\usebox{\alignbox}}
           \fi
          }%
          {\usebox{\alignbox}}}%
       \vspace*{\alignwidth}\pagebreak[0]\vspace{-\alignwidth}\par}
 % a multi-line comment, whose first argument is its width in points.


% \multispace{n} produces the vertical space indicated by "|"s in 
% this situation
%   
%     xxxx (*************)
%     xxxx (* ccccccccc *)
%      |   (* ccccccccc *)
%      |   (* ccccccccc *)
%      |   (* ccccccccc *)
%      |   (*************)
%
% where n is the number of "xxxx" lines.
\newcommand{\multivspace}[1]{\addtolength{\multicommentdepth}{-#1\baselineskip}%
 \addtolength{\multicommentdepth}{1.2em}%
 \ifthenelse{\lengthtest{\multicommentdepth > 0pt}}%
    {\par\vspace{\multicommentdepth}\par}{}}

%\newenvironment{hpar}[2]{%
%  \begin{list}{}{\setlength{\leftmargin}{#1pt}%
%                 \addtolength{\leftmargin}{#2pt}%
%                 \setlength{\itemindent}{-#2pt}%
%                 \setlength{\topsep}{0pt}%
%                 \setlength{\parsep}{0pt}%
%                 \setlength{\partopsep}{0pt}%
%                 \setlength{\parskip}{0pt}%
%                 \addtolength{\labelsep}{0pt}}%
%  \item[]\footnotesize}{\end{list}}
%    %%%%%%%%%%%%%%%%%%%%%%%%%%%%%%%%%%%%%%%%%%%%%%%%%%%%%%%%%%%%%%%%%%%%%%%%
%    % Typesets a sequence of paragraphs like this:                         %
%    %                                                                      %
%    %      left |<-- d1 --> XXXXXXXXXXXXXXXXXXXXXXXX                       %
%    %    margin |           <- d2 -> XXXXXXXXXXXXXXX                       %
%    %           |                    XXXXXXXXXXXXXXX                       %
%    %           |                                                          %
%    %           |                    XXXXXXXXXXXXXXX                       %
%    %           |                    XXXXXXXXXXXXXXX                       %
%    %                                                                      %
%    % where d1 = #1pt and d2 = #2pt, but with no vspace between            %
%    % paragraphs.                                                          %
%    %%%%%%%%%%%%%%%%%%%%%%%%%%%%%%%%%%%%%%%%%%%%%%%%%%%%%%%%%%%%%%%%%%%%%%%%

%%%%%%%%%%%%%%%%%%%%%%%%%%%%%%%%%%%%%%%%%%%%%%%%%%%%%%%%%%%%%%%%%%%%%%
% Commands for repeated characters that produce dashes.              %
%%%%%%%%%%%%%%%%%%%%%%%%%%%%%%%%%%%%%%%%%%%%%%%%%%%%%%%%%%%%%%%%%%%%%%
% \raisedDash{wd}{ht}{thk} makes a horizontal line wd characters wide, 
% raised a distance ht ex's above the baseline, with a thickness of 
% thk em's.
\newcommand{\raisedDash}[3]{\raisebox{#2ex}{\setlength{\alignwidth}{.5em}%
  \rule{#1\alignwidth}{#3em}}}

% The following commands take a single argument n and produce the
% output for n repeated characters, as follows
%   \cdash:    -
%   \tdash:    ~
%   \ceqdash:  =
%   \usdash:   _
\newcommand{\cdash}[1]{\raisedDash{#1}{.5}{.04}}
\newcommand{\usdash}[1]{\raisedDash{#1}{0}{.04}}
\newcommand{\ceqdash}[1]{\raisedDash{#1}{.5}{.08}}
\newcommand{\tdash}[1]{\raisedDash{#1}{1}{.08}}

\newlength{\spacewidth}
\setlength{\spacewidth}{.2em}
\newcommand{\e}[1]{\hspace{#1\spacewidth}}
%% \e{i} produces space corresponding to i input spaces.


%% Alignment-file Commands

\newlength{\alignboxwidth}
\newlength{\alignwidth}
\newsavebox{\alignbox}

% \al{i}{j}{txt} is used in the alignment file to put "%{i}{j}{wd}"
% in the log file, where wd is the width of the line up to that point,
% and txt is the following text.
\newcommand{\al}[3]{%
  \typeout{\%{#1}{#2}{\the\alignwidth}}%
  \cl{#3}}

%% \cl{txt} continues a specification line in the alignment file
%% with text txt.
\newcommand{\cl}[1]{%
  \savebox{\alignbox}{\mbox{$\mbox{}#1\mbox{}$}}%
  \settowidth{\alignboxwidth}{\usebox{\alignbox}}%
  \addtolength{\alignwidth}{\alignboxwidth}%
  \usebox{\alignbox}}

% \fl{txt} in the alignment file begins a specification line that
% starts with the text txt.
\newcommand{\fl}[1]{%
  \par
  \savebox{\alignbox}{\mbox{$\mbox{}#1\mbox{}$}}%
  \settowidth{\alignwidth}{\usebox{\alignbox}}%
  \usebox{\alignbox}}



  
%%%%%%%%%%%%%%%%%%%%%%%%%%%%%%%%%%%%%%%%%%%%%%%%%%%%%%%%%%%%%%%%%%%%%%%%%%%%%
% Ordinarily, TeX typesets letters in math mode in a special math italic    %
% font.  This makes it typeset "it" to look like the product of the         %
% variables i and t, rather than like the word "it".  The following         %
% commands tell TeX to use an ordinary italic font instead.                 %
%%%%%%%%%%%%%%%%%%%%%%%%%%%%%%%%%%%%%%%%%%%%%%%%%%%%%%%%%%%%%%%%%%%%%%%%%%%%%
\ifx\documentclass\undefined
\else
  \DeclareSymbolFont{tlaitalics}{\encodingdefault}{cmr}{m}{it}
  \let\itfam\symtlaitalics
\fi

\makeatletter
\newcommand{\tlx@c}{\c@tlx@ctr\advance\c@tlx@ctr\@ne}
\newcounter{tlx@ctr}
\c@tlx@ctr=\itfam \multiply\c@tlx@ctr"100\relax \advance\c@tlx@ctr "7061\relax
\mathcode`a=\tlx@c \mathcode`b=\tlx@c \mathcode`c=\tlx@c \mathcode`d=\tlx@c
\mathcode`e=\tlx@c \mathcode`f=\tlx@c \mathcode`g=\tlx@c \mathcode`h=\tlx@c
\mathcode`i=\tlx@c \mathcode`j=\tlx@c \mathcode`k=\tlx@c \mathcode`l=\tlx@c
\mathcode`m=\tlx@c \mathcode`n=\tlx@c \mathcode`o=\tlx@c \mathcode`p=\tlx@c
\mathcode`q=\tlx@c \mathcode`r=\tlx@c \mathcode`s=\tlx@c \mathcode`t=\tlx@c
\mathcode`u=\tlx@c \mathcode`v=\tlx@c \mathcode`w=\tlx@c \mathcode`x=\tlx@c
\mathcode`y=\tlx@c \mathcode`z=\tlx@c
\c@tlx@ctr=\itfam \multiply\c@tlx@ctr"100\relax \advance\c@tlx@ctr "7041\relax
\mathcode`A=\tlx@c \mathcode`B=\tlx@c \mathcode`C=\tlx@c \mathcode`D=\tlx@c
\mathcode`E=\tlx@c \mathcode`F=\tlx@c \mathcode`G=\tlx@c \mathcode`H=\tlx@c
\mathcode`I=\tlx@c \mathcode`J=\tlx@c \mathcode`K=\tlx@c \mathcode`L=\tlx@c
\mathcode`M=\tlx@c \mathcode`N=\tlx@c \mathcode`O=\tlx@c \mathcode`P=\tlx@c
\mathcode`Q=\tlx@c \mathcode`R=\tlx@c \mathcode`S=\tlx@c \mathcode`T=\tlx@c
\mathcode`U=\tlx@c \mathcode`V=\tlx@c \mathcode`W=\tlx@c \mathcode`X=\tlx@c
\mathcode`Y=\tlx@c \mathcode`Z=\tlx@c
\makeatother

%%%%%%%%%%%%%%%%%%%%%%%%%%%%%%%%%%%%%%%%%%%%%%%%%%%%%%%%%%
%                THE describe ENVIRONMENT                %
%%%%%%%%%%%%%%%%%%%%%%%%%%%%%%%%%%%%%%%%%%%%%%%%%%%%%%%%%%
%
%
% It is like the description environment except it takes an argument
% ARG that should be the text of the widest label.  It adjusts the
% indentation so each item with label LABEL produces
%%      LABEL             blah blah blah
%%      <- width of ARG ->blah blah blah
%%                        blah blah blah
\newenvironment{describe}[1]%
   {\begin{list}{}{\settowidth{\labelwidth}{#1}%
            \setlength{\labelsep}{.5em}%
            \setlength{\leftmargin}{\labelwidth}% 
            \addtolength{\leftmargin}{\labelsep}%
            \addtolength{\leftmargin}{\parindent}%
            \def\makelabel##1{\rm ##1\hfill}}%
            \setlength{\topsep}{0pt}}%% 
                % Sets \topsep to 0 to reduce vertical space above
                % and below embedded displayed equations
   {\end{list}}

%   For tlatex.TeX
\usepackage{verbatim}
\makeatletter
\def\tla{\let\%\relax%
         \@bsphack
         \typeout{\%{\the\linewidth}}%
             \let\do\@makeother\dospecials\catcode`\^^M\active
             \let\verbatim@startline\relax
             \let\verbatim@addtoline\@gobble
             \let\verbatim@processline\relax
             \let\verbatim@finish\relax
             \verbatim@}
\let\endtla=\@esphack

\let\pcal=\tla
\let\endpcal=\endtla
\let\ppcal=\tla
\let\endppcal=\endtla

% The tlatex environment is used by TLATeX.TeX to typeset TLA+.
% TLATeX.TLA starts its files by writing a \tlatex command.  This
% command/environment sets \parindent to 0 and defines \% to its
% standard definition because the writing of the log files is messed up
% if \% is defined to be something else.  It also executes
% \@computerule to determine the dimensions for the TLA horizonatl
% bars.
\newenvironment{tlatex}{\@computerule%
                        \setlength{\parindent}{0pt}%
                       \makeatletter\chardef\%=`\%}{}


% The notla environment produces no output.  You can turn a 
% tla environment to a notla environment to prevent tlatex.TeX from
% re-formatting the environment.

\def\notla{\let\%\relax%
         \@bsphack
             \let\do\@makeother\dospecials\catcode`\^^M\active
             \let\verbatim@startline\relax
             \let\verbatim@addtoline\@gobble
             \let\verbatim@processline\relax
             \let\verbatim@finish\relax
             \verbatim@}
\let\endnotla=\@esphack

\let\nopcal=\notla
\let\endnopcal=\endnotla
\let\noppcal=\notla
\let\endnoppcal=\endnotla

%%%%%%%%%%%%%%%%%%%%%%%% end of tlatex.sty file %%%%%%%%%%%%%%%%%%%%%%% 
% last modified on Fri  3 August 2012 at 14:23:49 PST by lamport

\begin{document}
\tlatex
\setboolean{shading}{true}
 \@x{\makebox[0pt][r]{\scriptsize 1\hspace{1em}}}\moduleLeftDash\@xx{
 {\MODULE} Sets}\moduleRightDash\@xx{}%
 \@x{\makebox[0pt][r]{\scriptsize 2\hspace{1em}} {\EXTENDS} Integers ,\,
 NaturalsInduction ,\, TLAPS}%
\@x{\makebox[0pt][r]{\scriptsize 3\hspace{1em}}\@s{8.2}}%
\@y{%
 \ensuremath{\.{*} NB}: Module \ensuremath{NaturalsInduction} comes from the
 \ensuremath{TLAPS} library, usually
}%
\@xx{}%
\@x{\makebox[0pt][r]{\scriptsize 4\hspace{1em}}\@s{8.2}}%
\@y{%
 * installed in /usr/local/lib/tlaps. Make sure this is in your
 \ensuremath{Toolbox
}}%
\@xx{}%
\@x{\makebox[0pt][r]{\scriptsize 5\hspace{1em}}\@s{8.2}}%
\@y{%
 * search path, see Preferences/TLA+ Preferences.
}%
\@xx{}%
\@pvspace{8.0pt}%
 \@x{\makebox[0pt][r]{\scriptsize 7\hspace{1em}} IsBijection ( f ,\, S ,\, T )
 \.{\defeq} \.{\land} f \.{\in} [ S \.{\rightarrow} T ]}%
 \@x{\makebox[0pt][r]{\scriptsize 8\hspace{1em}}\@s{107.07} \.{\land} \A\, x
 ,\, y \.{\in} S \.{:} ( x \.{\neq} y ) \.{\implies} ( f [ x ] \.{\neq} f [ y
 ] )}%
 \@x{\makebox[0pt][r]{\scriptsize 9\hspace{1em}}\@s{107.07} \.{\land} \A\, y
 \.{\in} T \.{:} \E\, x \.{\in} S \.{:} f [ x ] \.{=} y}%
\@pvspace{16.0pt}%
 \@x{\makebox[0pt][r]{\scriptsize 12\hspace{1em}} IsFiniteSet ( S ) \.{\defeq}
 \E\, n \.{\in} Nat \.{:} \E\, f\@s{6.33} \.{:} IsBijection ( f ,\, 1
 \.{\dotdot} n ,\, S )}%
\@pvspace{8.0pt}%
\begin{lcom}{0}%
\begin{cpar}{0}{F}{F}{0}{0}{}%
Finite sets and cardinality are defined in the TLA+ standard module
 \ensuremath{FiniteSets}, but this is not yet natively supported by
 \ensuremath{TLAPS}. For the
 time being, we use the following axiom for defining set cardinality.
\end{cpar}%
\end{lcom}%
\@x{\makebox[0pt][r]{\scriptsize 19\hspace{1em}}}%
\@y{\@s{0}%
 \ensuremath{Cardinality(S) \.{\defeq} {\CHOOSE} n} : (\ensuremath{n \.{\in}
 Nat}) \ensuremath{\.{\land} \E\, f} : \ensuremath{IsBijection(f,\,
 1\.{\dotdot}n,\, S)
}}%
\@xx{}%
\@pvspace{8.0pt}%
 \@x{\makebox[0pt][r]{\scriptsize 21\hspace{1em}} {\CONSTANT} Cardinality ( \_
 )}%
 \@x{\makebox[0pt][r]{\scriptsize 22\hspace{1em}} {\AXIOM} CardinalityAxiom
 \.{\defeq}}%
 \@x{\makebox[0pt][r]{\scriptsize 23\hspace{1em}}\@s{45.18} \A\, S \.{:}
 IsFiniteSet ( S ) \.{\implies}}%
 \@x{\makebox[0pt][r]{\scriptsize 24\hspace{1em}}\@s{53.38} \A\, n \.{:} ( n
 \.{=} Cardinality ( S ) ) \.{\equiv}}%
 \@x{\makebox[0pt][r]{\scriptsize 25\hspace{1em}}\@s{83.32} ( n \.{\in} Nat )
 \.{\land} \E\, f \.{:} IsBijection ( f ,\, 1 \.{\dotdot} n ,\, S )}%
\@x{\makebox[0pt][r]{\scriptsize 26\hspace{1em}}}\midbar\@xx{}%
\@pvspace{8.0pt}%
 \@x{\makebox[0pt][r]{\scriptsize 28\hspace{1em}} {\THEOREM} CardinalityInNat
 \.{\defeq} \A\, S \.{:} IsFiniteSet ( S ) \.{\implies} Cardinality ( S )
 \.{\in} Nat}%
\@x{\makebox[0pt][r]{\scriptsize 29\hspace{1em}} {\BY} CardinalityAxiom}%
\@pvspace{8.0pt}%
\@x{\makebox[0pt][r]{\scriptsize 31\hspace{1em}}}\midbar\@xx{}%
\@pvspace{8.0pt}%
 \@x{\makebox[0pt][r]{\scriptsize 33\hspace{1em}} {\THEOREM} CardinalityZero
 \.{\defeq}}%
 \@x{\makebox[0pt][r]{\scriptsize 34\hspace{1em}}\@s{59.12} \.{\land}
 IsFiniteSet ( \{ \} )}%
 \@x{\makebox[0pt][r]{\scriptsize 35\hspace{1em}}\@s{59.12} \.{\land}
 Cardinality ( \{ \} ) \.{=} 0}%
 \@x{\makebox[0pt][r]{\scriptsize 36\hspace{1em}}\@s{59.12} \.{\land} \A\, S
 \.{:} IsFiniteSet ( S ) \.{\land} ( Cardinality ( S ) \.{=} 0 ) \.{\implies}
 ( S \.{=} \{ \} )}%
 \@x{\makebox[0pt][r]{\scriptsize 37\hspace{1em}}\@pfstepnum{1}{1.} \.{\land}
 IsFiniteSet ( \{ \} )}%
 \@x{\makebox[0pt][r]{\scriptsize 38\hspace{1em}}\@s{20.55} \.{\land}
 Cardinality ( \{ \} ) \.{=} 0}%
 \@x{\makebox[0pt][r]{\scriptsize 39\hspace{1em}}\@s{8.2}\@pfstepnum{2}{1.}\ 
 IsBijection ( [ x \.{\in} 1 \.{\dotdot} 0 \.{\mapsto} \{ \} ] ,\, 1
 \.{\dotdot} 0 ,\, \{ \} )}%
 \@x{\makebox[0pt][r]{\scriptsize 40\hspace{1em}}\@s{16.4} {\BY} Z3 {\DEF}
 IsBijection}%
 \@x{\makebox[0pt][r]{\scriptsize 41\hspace{1em}}\@s{8.2}\@pfstepnum{2}{2.}\ 
 {\QED}}%
 \@x{\makebox[0pt][r]{\scriptsize 42\hspace{1em}}\@s{16.4}
 {\BY}\@pfstepnum{2}{1} ,\, CardinalityAxiom {\DEF} IsFiniteSet}%
 \@x{\makebox[0pt][r]{\scriptsize 43\hspace{1em}}\@pfstepnum{1}{2.}\ 
 {\ASSUME} {\NEW} S ,\,}%
 \@x{\makebox[0pt][r]{\scriptsize 44\hspace{1em}}\@s{62.13} IsFiniteSet ( S )
 ,\,}%
 \@x{\makebox[0pt][r]{\scriptsize 45\hspace{1em}}\@s{62.13} Cardinality ( S )
 \.{=} 0}%
 \@x{\makebox[0pt][r]{\scriptsize 46\hspace{1em}}\@s{23.88} {\PROVE}\@s{4.1} S
 \.{=} \{ \}}%
 \@x{\makebox[0pt][r]{\scriptsize 47\hspace{1em}}\@s{8.2}
 {\BY}\@pfstepnum{1}{2} ,\, CardinalityAxiom ,\, SMT {\DEF} IsBijection}%
\@x{\makebox[0pt][r]{\scriptsize 48\hspace{1em}}\@pfstepnum{1}{3.}\  {\QED}}%
 \@x{\makebox[0pt][r]{\scriptsize 49\hspace{1em}}\@s{8.2}
 {\BY}\@pfstepnum{1}{1} ,\,\@pfstepnum{1}{2}\ }%
\@pvspace{8.0pt}%
 \@x{\makebox[0pt][r]{\scriptsize 51\hspace{1em}} {\THEOREM}
 CardinalityPlusOne \.{\defeq}}%
 \@x{\makebox[0pt][r]{\scriptsize 52\hspace{1em}}\@s{16.4} {\ASSUME} {\NEW} S
 ,\, IsFiniteSet ( S ) ,\,}%
 \@x{\makebox[0pt][r]{\scriptsize 53\hspace{1em}}\@s{54.64} {\NEW} x ,\, x
 \.{\notin} S}%
 \@x{\makebox[0pt][r]{\scriptsize 54\hspace{1em}}\@s{16.4} {\PROVE}\@s{4.1}
 \.{\land} IsFiniteSet ( S\@s{0.92} \.{\cup} \{ x \} )}%
 \@x{\makebox[0pt][r]{\scriptsize 55\hspace{1em}}\@s{54.64} \.{\land}
 Cardinality ( S \.{\cup} \{ x \} ) \.{=} Cardinality ( S ) \.{+} 1}%
 \@x{\makebox[0pt][r]{\scriptsize 56\hspace{1em}}\@pfstepnum{1}{}\  {\DEFINE}
 N \.{\defeq} Cardinality ( S )}%
 \@x{\makebox[0pt][r]{\scriptsize 57\hspace{1em}}\@pfstepnum{1}{1.}\  {\PICK}
 f \.{:} IsBijection ( f ,\, 1 \.{\dotdot} N ,\, S )}%
 \@x{\makebox[0pt][r]{\scriptsize 58\hspace{1em}}\@s{8.2} {\BY}
 CardinalityAxiom}%
 \@x{\makebox[0pt][r]{\scriptsize 59\hspace{1em}}\@pfstepnum{1}{}\  {\DEFINE}
 g \.{\defeq} [ i \.{\in} 1 \.{\dotdot} ( N \.{+} 1 ) \.{\mapsto} {\IF} i
 \.{=} N \.{+} 1 \.{\THEN} x \.{\ELSE} f [ i ] ]}%
 \@x{\makebox[0pt][r]{\scriptsize 60\hspace{1em}}\@pfstepnum{1}{2.}\ 
 IsBijection ( g ,\, 1 \.{\dotdot} ( N \.{+} 1 ) ,\, S \.{\cup} \{ x \} )}%
 \@x{\makebox[0pt][r]{\scriptsize 61\hspace{1em}}\@s{8.2}
 {\BY}\@pfstepnum{1}{1} ,\, CardinalityInNat ,\, Z3 {\DEF} IsBijection}%
\@x{\makebox[0pt][r]{\scriptsize 62\hspace{1em}}\@pfstepnum{1}{3.}\  {\QED}}%
 \@x{\makebox[0pt][r]{\scriptsize 63\hspace{1em}}\@s{8.2}
 {\BY}\@pfstepnum{1}{2} ,\, CardinalityInNat ,\, CardinalityAxiom ,\, SMT
 {\DEF} IsFiniteSet}%
\@pvspace{8.0pt}%
\@x{\makebox[0pt][r]{\scriptsize 65\hspace{1em}}}\midbar\@xx{}%
\@pvspace{8.0pt}%
 \@x{\makebox[0pt][r]{\scriptsize 67\hspace{1em}} {\THEOREM} CardinalityOne
 \.{\defeq} \A\, m \.{:} \.{\land} IsFiniteSet ( \{ m \} )}%
 \@x{\makebox[0pt][r]{\scriptsize 68\hspace{1em}}\@s{157.53} \.{\land}
 Cardinality ( \{ m \} ) \.{=} 1}%
 \@x{\makebox[0pt][r]{\scriptsize 69\hspace{1em}} {\BY} CardinalityZero ,\,
 CardinalityPlusOne ,\, IsaM (\@w{auto} )}%
\@pvspace{8.0pt}%
 \@x{\makebox[0pt][r]{\scriptsize 71\hspace{1em}} {\THEOREM} CardinalityTwo
 \.{\defeq} \A\, m ,\, p \.{:} m \.{\neq} p \.{\implies}}%
 \@x{\makebox[0pt][r]{\scriptsize 72\hspace{1em}}\@s{150.10} \.{\land}
 IsFiniteSet ( \{ m ,\, p \} )}%
 \@x{\makebox[0pt][r]{\scriptsize 73\hspace{1em}}\@s{150.10} \.{\land}
 Cardinality ( \{ m ,\, p \} ) \.{=} 2}%
 \@x{\makebox[0pt][r]{\scriptsize 74\hspace{1em}} {\BY} CardinalityOne ,\,
 CardinalityPlusOne ,\, IsaM (\@w{auto} )}%
\@pvspace{8.0pt}%
 \@x{\makebox[0pt][r]{\scriptsize 76\hspace{1em}} {\THEOREM}
 IntervalCardinality \.{\defeq}}%
 \@x{\makebox[0pt][r]{\scriptsize 77\hspace{1em}}\@s{8.2} {\ASSUME} {\NEW} a
 \.{\in} Nat ,\, {\NEW} b \.{\in} Nat}%
 \@x{\makebox[0pt][r]{\scriptsize 78\hspace{1em}}\@s{8.2} {\PROVE}\@s{4.1}
 \.{\land} IsFiniteSet ( a \.{\dotdot} b )}%
 \@x{\makebox[0pt][r]{\scriptsize 79\hspace{1em}}\@s{46.44} \.{\land}
 Cardinality ( a \.{\dotdot} b ) \.{=} {\IF} a \.{>} b \.{\THEN} 0 \.{\ELSE}
 b \.{-} a \.{+} 1}%
 \@x{\makebox[0pt][r]{\scriptsize 80\hspace{1em}}\@pfstepnum{1}{1.} {\CASE} a
 \.{>} b}%
 \@x{\makebox[0pt][r]{\scriptsize 81\hspace{1em}}\@s{8.2}
 {\BY}\@pfstepnum{1}{1} ,\, CardinalityZero ,\, a \.{\dotdot} b \.{=} \{ \}
 ,\, IsFiniteSet ( a \.{\dotdot} b ) ,\,}%
 \@x{\makebox[0pt][r]{\scriptsize 82\hspace{1em}}\@s{23.91} Cardinality ( a
 \.{\dotdot} b ) \.{=} 0 ,\, SMT}%
 \@x{\makebox[0pt][r]{\scriptsize 83\hspace{1em}}\@pfstepnum{1}{2.} {\CASE} a
 \.{\leq} b}%
 \@x{\makebox[0pt][r]{\scriptsize 84\hspace{1em}}\@s{8.2}\@pfstepnum{2}{}\ 
 {\DEFINE} n\@s{1.46} \.{\defeq} b \.{-} a \.{+} 1}%
 \@x{\makebox[0pt][r]{\scriptsize 85\hspace{1em}}\@s{8.2}\@pfstepnum{2}{}\ 
 {\DEFINE} F \.{\defeq} [ x \.{\in} 1 \.{\dotdot} n \.{\mapsto} x \.{+} a
 \.{-} 1 ]}%
 \@x{\makebox[0pt][r]{\scriptsize 86\hspace{1em}}\@s{8.2}\@pfstepnum{2}{1.}\ 
 \A\, y \.{\in} a \.{\dotdot} b \.{:}\@s{4.1} \E\, x \.{\in} 1 \.{\dotdot} n
 \.{:} y \.{+} 1 \.{-} a \.{=} x}%
\begin{lcom}{15.0}%
\begin{cpar}{0}{T}{T}{0}{2.5}{%
*}%
This equation cannot be proved by \ensuremath{SMTs} if the variables
 are in a different order.
\end{cpar}%
\end{lcom}%
 \@x{\makebox[0pt][r]{\scriptsize 89\hspace{1em}}\@s{16.4}
 {\BY}\@pfstepnum{1}{2} ,\, SMT}%
 \@x{\makebox[0pt][r]{\scriptsize 90\hspace{1em}}\@s{8.2}\@pfstepnum{2}{2.}\ 
 IsBijection ( F ,\, 1 \.{\dotdot} n ,\, a \.{\dotdot} b )}%
 \@x{\makebox[0pt][r]{\scriptsize 91\hspace{1em}}\@s{16.4}
 {\BY}\@pfstepnum{2}{1} ,\, Z3 {\DEF} IsBijection}%
 \@x{\makebox[0pt][r]{\scriptsize 92\hspace{1em}}\@s{8.2}\@pfstepnum{2}{}\ 
 {\QED}}%
 \@x{\makebox[0pt][r]{\scriptsize 93\hspace{1em}}\@s{16.4}
 {\BY}\@pfstepnum{2}{2} ,\,\@pfstepnum{1}{2} ,\, CardinalityAxiom ,\, SMT
 {\DEF} IsFiniteSet}%
\@x{\makebox[0pt][r]{\scriptsize 94\hspace{1em}}\@pfstepnum{1}{q.}\  {\QED}}%
 \@x{\makebox[0pt][r]{\scriptsize 95\hspace{1em}}\@s{8.2}
 {\BY}\@pfstepnum{1}{1} ,\,\@pfstepnum{1}{2} ,\, SMT}%
\@pvspace{8.0pt}%
\@x{\makebox[0pt][r]{\scriptsize 97\hspace{1em}}}\midbar\@xx{}%
\@pvspace{8.0pt}%
 \@x{\makebox[0pt][r]{\scriptsize 99\hspace{1em}} {\THEOREM}
 CardinalityOneConverse \.{\defeq}}%
 \@x{\makebox[0pt][r]{\scriptsize 100\hspace{1em}}\@s{12.29} {\ASSUME} {\NEW}
 S ,\, IsFiniteSet ( S ) ,\, Cardinality ( S ) \.{=} 1}%
 \@x{\makebox[0pt][r]{\scriptsize 101\hspace{1em}}\@s{12.29} {\PROVE}\@s{4.1}
 \E\, m \.{:} S \.{=} \{ m \}}%
 \@x{\makebox[0pt][r]{\scriptsize 102\hspace{1em}}\@pfstepnum{1}{1.}\  {\PICK}
 f \.{:} IsBijection ( f ,\, 1 \.{\dotdot} 1 ,\, S )}%
 \@x{\makebox[0pt][r]{\scriptsize 103\hspace{1em}}\@s{8.2} {\BY}
 CardinalityAxiom}%
 \@x{\makebox[0pt][r]{\scriptsize 104\hspace{1em}}\@pfstepnum{1}{2.}\  S \.{=}
 \{ f [ 1 ] \}}%
 \@x{\makebox[0pt][r]{\scriptsize 105\hspace{1em}}\@s{8.2}
 {\BY}\@pfstepnum{1}{1} ,\, SMT {\DEF} IsBijection}%
\@x{\makebox[0pt][r]{\scriptsize 106\hspace{1em}}\@pfstepnum{1}{q.}\  {\QED}}%
 \@x{\makebox[0pt][r]{\scriptsize 107\hspace{1em}}\@s{8.2}
 {\BY}\@pfstepnum{1}{2}\ }%
\@pvspace{8.0pt}%
\@x{\makebox[0pt][r]{\scriptsize 109\hspace{1em}}}\midbar\@xx{}%
\@pvspace{8.0pt}%
 \@x{\makebox[0pt][r]{\scriptsize 111\hspace{1em}} {\THEOREM}
 IsBijectionInverse \.{\defeq}}%
 \@x{\makebox[0pt][r]{\scriptsize 112\hspace{1em}}\@s{8.2} {\ASSUME} {\NEW} f
 ,\, {\NEW} S ,\, {\NEW} T ,\,}%
 \@x{\makebox[0pt][r]{\scriptsize 113\hspace{1em}}\@s{46.44} IsBijection ( f
 ,\, S ,\, T )}%
 \@x{\makebox[0pt][r]{\scriptsize 114\hspace{1em}}\@s{8.2} {\PROVE}\@s{4.1}
 \E\, g \.{:} IsBijection ( g ,\, T ,\, S )}%
 \@x{\makebox[0pt][r]{\scriptsize 115\hspace{1em}}\@pfstepnum{1}{}\ 
 {\WITNESS} [ y \.{\in} T \.{\mapsto} {\CHOOSE} x \.{\in} S \.{:} f [ x ]
 \.{=} y ]}%
\@x{\makebox[0pt][r]{\scriptsize 116\hspace{1em}}\@pfstepnum{1}{}\  {\QED}}%
 \@x{\makebox[0pt][r]{\scriptsize 117\hspace{1em}}\@s{8.2} {\BY} {\DEF}
 IsBijection}%
\@pvspace{8.0pt}%
 \@x{\makebox[0pt][r]{\scriptsize 119\hspace{1em}} {\THEOREM}
 IsBijectionTransitive \.{\defeq}}%
 \@x{\makebox[0pt][r]{\scriptsize 120\hspace{1em}}\@s{8.2} {\ASSUME} {\NEW} f1
 ,\, {\NEW} f2 ,\, {\NEW} S ,\, {\NEW} T ,\, {\NEW} U ,\,}%
 \@x{\makebox[0pt][r]{\scriptsize 121\hspace{1em}}\@s{54.64} IsBijection ( f1
 ,\, S ,\, U ) ,\,}%
 \@x{\makebox[0pt][r]{\scriptsize 122\hspace{1em}}\@s{54.64} IsBijection ( f2
 ,\, U ,\, T )}%
 \@x{\makebox[0pt][r]{\scriptsize 123\hspace{1em}}\@s{8.2} {\PROVE}\@s{4.1}
 \E\, g \.{:} IsBijection ( g ,\, S ,\, T )}%
 \@x{\makebox[0pt][r]{\scriptsize 124\hspace{1em}}\@pfstepnum{1}{}\ 
 {\WITNESS} [ x \.{\in} S \.{\mapsto} f2 [ f1 [ x ] ] ]}%
\@x{\makebox[0pt][r]{\scriptsize 125\hspace{1em}}\@pfstepnum{1}{}\  {\QED}}%
 \@x{\makebox[0pt][r]{\scriptsize 126\hspace{1em}}\@s{8.2} {\BY} SMT {\DEF}
 IsBijection}%
\@pvspace{8.0pt}%
\@x{\makebox[0pt][r]{\scriptsize 128\hspace{1em}} {\THEOREM}}%
 \@x{\makebox[0pt][r]{\scriptsize 129\hspace{1em}}\@s{16.4} {\ASSUME} {\NEW} n
 \.{\in} Nat ,\, {\NEW} m \.{\in} Nat ,\,}%
 \@x{\makebox[0pt][r]{\scriptsize 130\hspace{1em}}\@s{54.64} IsBijection ( [ x
 \.{\in} 1 \.{\dotdot} n \.{\mapsto} x ] ,\, 1 \.{\dotdot} n ,\, 1
 \.{\dotdot} m )}%
 \@x{\makebox[0pt][r]{\scriptsize 131\hspace{1em}}\@s{16.4} {\PROVE}\@s{4.1} n
 \.{=} m}%
\@pvspace{8.0pt}%
 \@x{\makebox[0pt][r]{\scriptsize 133\hspace{1em}} {\THEOREM}
 IsBijectionCardinality \.{\defeq}}%
 \@x{\makebox[0pt][r]{\scriptsize 134\hspace{1em}}\@s{8.2} \A\, f ,\, S ,\, T
 \.{:} \.{\land} IsFiniteSet ( S )}%
 \@x{\makebox[0pt][r]{\scriptsize 135\hspace{1em}}\@s{56.46} \.{\land}
 IsFiniteSet ( T )}%
 \@x{\makebox[0pt][r]{\scriptsize 136\hspace{1em}}\@s{56.46} \.{\implies} (
 IsBijection ( f ,\, S ,\, T ) \.{\equiv} Cardinality ( S ) \.{=} Cardinality
 ( T ) )}%
\@pvspace{8.0pt}%
 \@x{\makebox[0pt][r]{\scriptsize 138\hspace{1em}} {\LEMMA}
 CardinalitySetMinus \.{\defeq}}%
 \@x{\makebox[0pt][r]{\scriptsize 139\hspace{1em}}\@s{35.55} {\ASSUME} {\NEW}
 S ,\, IsFiniteSet ( S ) ,\,}%
 \@x{\makebox[0pt][r]{\scriptsize 140\hspace{1em}}\@s{73.79} {\NEW} x \.{\in}
 S}%
 \@x{\makebox[0pt][r]{\scriptsize 141\hspace{1em}}\@s{35.55} {\PROVE}
 \.{\land} IsFiniteSet ( S\@s{0.92} \.{\,\backslash\,} \{ x \} )}%
 \@x{\makebox[0pt][r]{\scriptsize 142\hspace{1em}}\@s{69.69} \.{\land}
 Cardinality ( S \.{\,\backslash\,} \{ x \} ) \.{=} Cardinality ( S ) \.{-}
 1}%
 \@x{\makebox[0pt][r]{\scriptsize 143\hspace{1em}}\@pfstepnum{1}{}\  {\DEFINE}
 N \.{\defeq} Cardinality ( S )}%
 \@x{\makebox[0pt][r]{\scriptsize 144\hspace{1em}}\@pfstepnum{1}{1.}\ 
 IsFiniteSet ( S \.{\,\backslash\,} \{ x \} )}%
 \@x{\makebox[0pt][r]{\scriptsize 145\hspace{1em}}\@s{8.2}\@pfstepnum{2}{g.}\ 
 {\PICK} g \.{:} IsBijection ( g ,\, 1 \.{\dotdot} N ,\, S )}%
 \@x{\makebox[0pt][r]{\scriptsize 146\hspace{1em}}\@s{16.4} {\BY}
 CardinalityAxiom}%
 \@x{\makebox[0pt][r]{\scriptsize 147\hspace{1em}}\@s{8.2}\@pfstepnum{2}{k.}\ 
 {\PICK} k \.{\in} 1 \.{\dotdot} N \.{:} g [ k ] \.{=} x}%
 \@x{\makebox[0pt][r]{\scriptsize 148\hspace{1em}}\@s{16.4}
 {\BY}\@pfstepnum{2}{g}\  {\DEF} IsBijection}%
 \@x{\makebox[0pt][r]{\scriptsize 149\hspace{1em}}\@s{8.2}\@pfstepnum{2}{}
 \.{\land} N \.{\in} Nat}%
 \@x{\makebox[0pt][r]{\scriptsize 150\hspace{1em}}\@s{20.97} \.{\land} N \.{-}
 1 \.{\in} Nat}%
 \@x{\makebox[0pt][r]{\scriptsize 151\hspace{1em}}\@s{16.4} {\BY}
 CardinalityInNat ,\, CardinalityZero ,\, SMT}%
 \@x{\makebox[0pt][r]{\scriptsize 152\hspace{1em}}\@s{8.2}\@pfstepnum{2}{}\ 
 {\DEFINE} f \.{\defeq} [ i \.{\in} 1 \.{\dotdot} N \.{-} 1 \.{\mapsto} g [
 {\IF} i \.{<} k \.{\THEN} i \.{\ELSE} i \.{+} 1 ] ]}%
 \@x{\makebox[0pt][r]{\scriptsize 153\hspace{1em}}\@s{8.2}\@pfstepnum{2}{}\ 
 {\HIDE} {\DEF} f}%
 \@x{\makebox[0pt][r]{\scriptsize 154\hspace{1em}}\@s{8.2}\@pfstepnum{2}{}\ 
 {\SUFFICES} IsBijection ( f ,\, 1 \.{\dotdot} N \.{-} 1 ,\, S\@s{4.1}
 \.{\,\backslash\,}\@s{4.1} \{ x \} )}%
 \@x{\makebox[0pt][r]{\scriptsize 155\hspace{1em}}\@s{16.4} {\BY} {\DEF}
 IsFiniteSet}%
 \@x{\makebox[0pt][r]{\scriptsize 156\hspace{1em}}\@s{8.2}\@pfstepnum{2}{1.}\ 
 f \.{\in} [ 1 \.{\dotdot} N \.{-} 1 \.{\rightarrow} S \.{\,\backslash\,} \{
 x \} ]}%
 \@x{\makebox[0pt][r]{\scriptsize 157\hspace{1em}}\@s{16.4}
 {\BY}\@pfstepnum{2}{g} ,\,\@pfstepnum{2}{k} ,\, SMT {\DEF} IsBijection ,\,
 f}%
 \@x{\makebox[0pt][r]{\scriptsize 158\hspace{1em}}\@s{8.2}\@pfstepnum{2}{2.}\ 
 {\ASSUME} {\NEW} i\@s{0.42} \.{\in} 1 \.{\dotdot} N \.{-} 1 ,\,}%
 \@x{\makebox[0pt][r]{\scriptsize 159\hspace{1em}}\@s{70.33} {\NEW} j \.{\in}
 1 \.{\dotdot} N \.{-} 1 ,\,}%
\@x{\makebox[0pt][r]{\scriptsize 160\hspace{1em}}\@s{70.33} i \.{\neq} j}%
 \@x{\makebox[0pt][r]{\scriptsize 161\hspace{1em}}\@s{32.08} {\PROVE}\@s{4.1}
 f [ i ] \.{\neq} f [ j ]}%
 \@x{\makebox[0pt][r]{\scriptsize 162\hspace{1em}}\@s{16.4}
 {\BY}\@pfstepnum{2}{g} ,\,\@pfstepnum{2}{2} ,\, SMTT ( 30 ) {\DEF}
 IsBijection ,\, f}%
 \@x{\makebox[0pt][r]{\scriptsize 163\hspace{1em}}\@s{8.2}\@pfstepnum{2}{3.}\ 
 {\ASSUME} {\NEW} y \.{\in} S \.{\,\backslash\,} \{ x \}}%
 \@x{\makebox[0pt][r]{\scriptsize 164\hspace{1em}}\@s{32.08} {\PROVE}\@s{4.1}
 \E\, i \.{\in} 1 \.{\dotdot} N \.{-} 1 \.{:} f [ i ] \.{=} y}%
 \@x{\makebox[0pt][r]{\scriptsize 165\hspace{1em}}\@s{16.4}\@pfstepnum{3}{j.}\
 {\PICK} j \.{\in} 1 \.{\dotdot} N \.{:} g [ j ] \.{=} y}%
 \@x{\makebox[0pt][r]{\scriptsize 166\hspace{1em}}\@s{24.59}
 {\BY}\@pfstepnum{2}{g}\  {\DEF} IsBijection}%
 \@x{\makebox[0pt][r]{\scriptsize 167\hspace{1em}}\@s{16.4}\@pfstepnum{3}{1.}
 {\CASE} j \.{<} k}%
 \@x{\makebox[0pt][r]{\scriptsize 168\hspace{1em}}\@s{24.59}
 {\BY}\@pfstepnum{3}{j} ,\,\@pfstepnum{3}{1} ,\, Z3 {\DEF} f}%
 \@x{\makebox[0pt][r]{\scriptsize 169\hspace{1em}}\@s{16.4}\@pfstepnum{3}{2.}
 {\CASE} {\lnot} ( j\@s{2.77} \.{<} k )}%
 \@x{\makebox[0pt][r]{\scriptsize 170\hspace{1em}}\@s{24.59}\@pfstepnum{4}{}
 \.{\land} {\lnot} ( j \.{-} 1 \.{<} k )}%
 \@x{\makebox[0pt][r]{\scriptsize 171\hspace{1em}}\@s{37.37} \.{\land} ( j
 \.{-} 1 ) \.{+} 1 \.{=} j}%
 \@x{\makebox[0pt][r]{\scriptsize 172\hspace{1em}}\@s{37.37} \.{\land} j \.{-}
 1 \.{\in} 1 \.{\dotdot} N \.{-} 1}%
 \@x{\makebox[0pt][r]{\scriptsize 173\hspace{1em}}\@s{32.8}
 {\BY}\@pfstepnum{3}{j} ,\,\@pfstepnum{3}{2} ,\,\@pfstepnum{2}{k} ,\, SMT}%
 \@x{\makebox[0pt][r]{\scriptsize 174\hspace{1em}}\@s{24.59}\@pfstepnum{4}{}\ 
 {\QED}}%
 \@x{\makebox[0pt][r]{\scriptsize 175\hspace{1em}}\@s{32.8}
 {\BY}\@pfstepnum{3}{j}\  {\DEF} f}%
 \@x{\makebox[0pt][r]{\scriptsize 176\hspace{1em}}\@s{16.4}\@pfstepnum{3}{4.}\
 {\QED}}%
 \@x{\makebox[0pt][r]{\scriptsize 177\hspace{1em}}\@s{24.59}
 {\BY}\@pfstepnum{3}{1} ,\,\@pfstepnum{3}{2}\ }%
 \@x{\makebox[0pt][r]{\scriptsize 178\hspace{1em}}\@s{8.2}\@pfstepnum{2}{q.}\ 
 {\QED}}%
 \@x{\makebox[0pt][r]{\scriptsize 179\hspace{1em}}\@s{16.4}
 {\BY}\@pfstepnum{2}{1} ,\,\@pfstepnum{2}{2} ,\,\@pfstepnum{2}{3}\  {\DEF}
 IsBijection}%
 \@x{\makebox[0pt][r]{\scriptsize 180\hspace{1em}}\@pfstepnum{1}{2.}\ 
 Cardinality ( S \.{\,\backslash\,} \{ x \} ) \.{=} Cardinality ( S ) \.{-}
 1}%
 \@x{\makebox[0pt][r]{\scriptsize 181\hspace{1em}}\@s{8.2} {\PROOF}
 {\OMITTED}}%
\@x{\makebox[0pt][r]{\scriptsize 182\hspace{1em}}\@pfstepnum{1}{q.}\  {\QED}}%
 \@x{\makebox[0pt][r]{\scriptsize 183\hspace{1em}}\@s{8.2}
 {\BY}\@pfstepnum{1}{1} ,\,\@pfstepnum{1}{2}\ }%
\@pvspace{8.0pt}%
 \@x{\makebox[0pt][r]{\scriptsize 185\hspace{1em}} {\THEOREM} FiniteSubset
 \.{\defeq}}%
 \@x{\makebox[0pt][r]{\scriptsize 186\hspace{1em}}\@s{8.2} {\ASSUME} {\NEW} S
 ,\, {\NEW} TT ,\, IsFiniteSet ( TT ) ,\, S \.{\subseteq} TT}%
 \@x{\makebox[0pt][r]{\scriptsize 187\hspace{1em}}\@s{8.2} {\PROVE}\@s{4.1}
 \.{\land} IsFiniteSet ( S )}%
 \@x{\makebox[0pt][r]{\scriptsize 188\hspace{1em}}\@s{46.44} \.{\land}
 Cardinality ( S ) \.{\leq} Cardinality ( TT )}%
 \@x{\makebox[0pt][r]{\scriptsize 189\hspace{1em}}\@pfstepnum{1}{2.}\  {\PICK}
 N \.{\in} Nat \.{:} N \.{=} Cardinality ( TT )}%
 \@x{\makebox[0pt][r]{\scriptsize 190\hspace{1em}}\@s{8.2} {\BY}
 CardinalityAxiom}%
 \@x{\makebox[0pt][r]{\scriptsize 191\hspace{1em}}\@pfstepnum{1}{3.}\ \@s{4.1}
 IsFiniteSet ( S )}%
 \@x{\makebox[0pt][r]{\scriptsize 192\hspace{1em}}\@s{8.2}\@pfstepnum{2}{}\ 
 {\DEFINE} P ( n ) \.{\defeq} \A\, T \.{:} S \.{\subseteq} T \.{\land}
 IsFiniteSet ( T ) \.{\land} Cardinality ( T ) \.{=} n}%
 \@x{\makebox[0pt][r]{\scriptsize 193\hspace{1em}}\@s{124.90} \.{\implies}
 IsFiniteSet ( S )}%
 \@x{\makebox[0pt][r]{\scriptsize 194\hspace{1em}}\@s{8.2}\@pfstepnum{2}{2.}\ 
 P ( 0 )}%
 \@x{\makebox[0pt][r]{\scriptsize 195\hspace{1em}}\@s{16.4} {\BY}
 CardinalityZero}%
 \@x{\makebox[0pt][r]{\scriptsize 196\hspace{1em}}\@s{8.2}\@pfstepnum{2}{3.}\ 
 {\ASSUME} {\NEW} n \.{\in} Nat ,\, P ( n )}%
 \@x{\makebox[0pt][r]{\scriptsize 197\hspace{1em}}\@s{32.08} {\PROVE}\@s{4.1}
 P ( n \.{+} 1 )}%
 \@x{\makebox[0pt][r]{\scriptsize 198\hspace{1em}}\@s{16.4}\@pfstepnum{3}{1.}\
 {\SUFFICES} {\ASSUME} \A\, R \.{:} \.{\land} S \.{\subseteq} R}%
 \@x{\makebox[0pt][r]{\scriptsize 199\hspace{1em}}\@s{146.05} \.{\land}
 IsFiniteSet ( R )}%
 \@x{\makebox[0pt][r]{\scriptsize 200\hspace{1em}}\@s{146.05} \.{\land}
 Cardinality ( R ) \.{=} n}%
 \@x{\makebox[0pt][r]{\scriptsize 201\hspace{1em}}\@s{146.05} \.{\implies}
 IsFiniteSet ( S ) ,\,}%
\@x{\makebox[0pt][r]{\scriptsize 202\hspace{1em}}\@s{122.82} {\NEW} T ,\,}%
 \@x{\makebox[0pt][r]{\scriptsize 203\hspace{1em}}\@s{122.82} S \.{\subseteq}
 T ,\,}%
 \@x{\makebox[0pt][r]{\scriptsize 204\hspace{1em}}\@s{122.82} IsFiniteSet ( T
 ) ,\,}%
 \@x{\makebox[0pt][r]{\scriptsize 205\hspace{1em}}\@s{122.82} Cardinality ( T
 ) \.{=} n \.{+} 1 ,\,}%
 \@x{\makebox[0pt][r]{\scriptsize 206\hspace{1em}}\@s{122.82} {\NEW} x \.{\in}
 T ,\, x\@s{2.37} \.{\notin} S}%
 \@x{\makebox[0pt][r]{\scriptsize 207\hspace{1em}}\@s{84.57} {\PROVE}\@s{4.1}
 IsFiniteSet ( S )}%
 \@x{\makebox[0pt][r]{\scriptsize 208\hspace{1em}}\@s{20.5}
 {\BY}\@pfstepnum{2}{3} ,\, SetExtensionality ,\, SMT}%
 \@x{\makebox[0pt][r]{\scriptsize
 209\hspace{1em}}\@s{12.29}\@pfstepnum{3}{2.}\  IsFiniteSet ( T
 \.{\,\backslash\,} \{ x \} )}%
 \@x{\makebox[0pt][r]{\scriptsize 210\hspace{1em}}\@s{20.5}
 {\BY}\@pfstepnum{3}{1} ,\, CardinalitySetMinus}%
 \@x{\makebox[0pt][r]{\scriptsize
 211\hspace{1em}}\@s{12.29}\@pfstepnum{3}{q.}\  {\QED}}%
 \@x{\makebox[0pt][r]{\scriptsize 212\hspace{1em}}\@s{20.5}
 {\BY}\@pfstepnum{3}{1} ,\,\@pfstepnum{3}{2} ,\, CardinalityPlusOne ,\,
 CardinalityInNat ,\, SMT}%
 \@x{\makebox[0pt][r]{\scriptsize 213\hspace{1em}}\@s{8.2}\@pfstepnum{2}{} .
 {\HIDE} {\DEF} P}%
 \@x{\makebox[0pt][r]{\scriptsize 214\hspace{1em}}\@s{8.2}\@pfstepnum{2}{4.}\ 
 \A\, n \.{\in} Nat \.{:} P ( n )}%
 \@x{\makebox[0pt][r]{\scriptsize 215\hspace{1em}}\@s{16.4}
 {\BY}\@pfstepnum{1}{2} ,\,\@pfstepnum{2}{2} ,\,\@pfstepnum{2}{3} ,\,
 NatInduction}%
 \@x{\makebox[0pt][r]{\scriptsize 216\hspace{1em}}\@s{8.2}\@pfstepnum{2}{q.}\ 
 {\QED}}%
 \@x{\makebox[0pt][r]{\scriptsize 217\hspace{1em}}\@s{16.4}
 {\BY}\@pfstepnum{1}{2} ,\,\@pfstepnum{2}{4}\  {\DEF} P}%
 \@x{\makebox[0pt][r]{\scriptsize 218\hspace{1em}}\@pfstepnum{1}{4.}\ 
 Cardinality ( S ) \.{\leq} Cardinality ( TT )}%
 \@x{\makebox[0pt][r]{\scriptsize 219\hspace{1em}}\@s{8.2}\@pfstepnum{2}{}\ 
 {\DEFINE} P ( n ) \.{\defeq} \A\, T \.{:} \.{\land} S \.{\subseteq} T}%
 \@x{\makebox[0pt][r]{\scriptsize 220\hspace{1em}}\@s{124.90} \.{\land}
 IsFiniteSet ( T )}%
 \@x{\makebox[0pt][r]{\scriptsize 221\hspace{1em}}\@s{124.90} \.{\land}
 IsFiniteSet ( S )}%
 \@x{\makebox[0pt][r]{\scriptsize 222\hspace{1em}}\@s{124.90} \.{\land}
 Cardinality ( T )\@s{2.77} \.{=} n}%
 \@x{\makebox[0pt][r]{\scriptsize 223\hspace{1em}}\@s{124.90} \.{\implies}
 Cardinality ( S ) \.{\leq} Cardinality ( T )}%
 \@x{\makebox[0pt][r]{\scriptsize 224\hspace{1em}}\@s{8.2}\@pfstepnum{2}{1.}\ 
 P ( 0 )}%
 \@x{\makebox[0pt][r]{\scriptsize 225\hspace{1em}}\@s{16.4} {\BY}
 CardinalityZero ,\, SetExtensionality ,\, SMT}%
 \@x{\makebox[0pt][r]{\scriptsize 226\hspace{1em}}\@s{8.2}\@pfstepnum{2}{2.}\ 
 {\ASSUME} {\NEW} n \.{\in} Nat ,\, P ( n )}%
 \@x{\makebox[0pt][r]{\scriptsize 227\hspace{1em}}\@s{32.08} {\PROVE}\@s{4.1}
 P ( n \.{+} 1 )}%
 \@x{\makebox[0pt][r]{\scriptsize 228\hspace{1em}}\@s{16.4}\@pfstepnum{3}{}\ 
 {\SUFFICES} {\ASSUME} \A\, R \.{:}}%
 \@x{\makebox[0pt][r]{\scriptsize 229\hspace{1em}}\@s{122.26} \.{\land}
 S\@s{4.1} \.{\subseteq}\@s{4.1} R}%
 \@x{\makebox[0pt][r]{\scriptsize 230\hspace{1em}}\@s{122.26} \.{\land}
 IsFiniteSet ( R )}%
 \@x{\makebox[0pt][r]{\scriptsize 231\hspace{1em}}\@s{122.26} \.{\land}
 IsFiniteSet ( S )}%
 \@x{\makebox[0pt][r]{\scriptsize 232\hspace{1em}}\@s{122.26} \.{\land}
 Cardinality ( R )\@s{4.1} \.{=}\@s{4.1} n}%
 \@x{\makebox[0pt][r]{\scriptsize 233\hspace{1em}}\@s{122.26}
 \.{\implies}\@s{4.1} Cardinality ( S )\@s{4.1} \.{\leq}\@s{4.1} Cardinality
 ( R ) ,\,}%
\@x{\makebox[0pt][r]{\scriptsize 234\hspace{1em}}\@s{115.04} {\NEW} T ,\,}%
 \@x{\makebox[0pt][r]{\scriptsize 235\hspace{1em}}\@s{115.04} S\@s{4.1}
 \.{\subseteq}\@s{4.1} T ,\,}%
 \@x{\makebox[0pt][r]{\scriptsize 236\hspace{1em}}\@s{115.04} IsFiniteSet ( T
 ) ,\,}%
 \@x{\makebox[0pt][r]{\scriptsize 237\hspace{1em}}\@s{115.04} IsFiniteSet ( S
 ) ,\,}%
 \@x{\makebox[0pt][r]{\scriptsize 238\hspace{1em}}\@s{115.04} Cardinality ( T
 )\@s{4.1} \.{=}\@s{4.1} n \.{+} 1 ,\,}%
 \@x{\makebox[0pt][r]{\scriptsize 239\hspace{1em}}\@s{115.04} {\NEW} x \.{\in}
 T ,\, x \.{\notin} S}%
 \@x{\makebox[0pt][r]{\scriptsize 240\hspace{1em}}\@s{65.31} {\PROVE}
 Cardinality ( S )\@s{4.1} \.{\leq}\@s{4.1} Cardinality ( T )}%
 \@x{\makebox[0pt][r]{\scriptsize 241\hspace{1em}}\@s{24.59}
 {\BY}\@pfstepnum{2}{2} ,\, SetExtensionality ,\, SMT}%
 \@x{\makebox[0pt][r]{\scriptsize 242\hspace{1em}}\@s{16.4}\@pfstepnum{3}{}
 \.{\land} IsFiniteSet ( T\@s{0.92} \.{\,\backslash\,} \{ x \} )}%
 \@x{\makebox[0pt][r]{\scriptsize 243\hspace{1em}}\@s{29.17} \.{\land}
 Cardinality ( T \.{\,\backslash\,} \{ x \} ) \.{=} Cardinality ( T ) \.{-}
 1}%
 \@x{\makebox[0pt][r]{\scriptsize 244\hspace{1em}}\@s{24.59} {\BY}
 CardinalitySetMinus}%
 \@x{\makebox[0pt][r]{\scriptsize 245\hspace{1em}}\@s{16.4}\@pfstepnum{3}{}\ 
 {\QED}}%
 \@x{\makebox[0pt][r]{\scriptsize 246\hspace{1em}}\@s{24.59} {\BY}
 CardinalityPlusOne ,\, CardinalityInNat ,\, Z3}%
 \@x{\makebox[0pt][r]{\scriptsize 247\hspace{1em}}\@s{8.2}\@pfstepnum{2}{}\ 
 {\HIDE} {\DEF} P}%
 \@x{\makebox[0pt][r]{\scriptsize 248\hspace{1em}}\@s{8.2}\@pfstepnum{2}{3.}\ 
 \A\, n \.{\in} Nat \.{:} P ( n )}%
 \@x{\makebox[0pt][r]{\scriptsize 249\hspace{1em}}\@s{16.4}
 {\BY}\@pfstepnum{1}{2} ,\,\@pfstepnum{2}{1} ,\,\@pfstepnum{2}{2} ,\,
 NatInduction}%
 \@x{\makebox[0pt][r]{\scriptsize 250\hspace{1em}}\@s{8.2}\@pfstepnum{2}{q.}\ 
 {\QED}}%
 \@x{\makebox[0pt][r]{\scriptsize 251\hspace{1em}}\@s{16.4}
 {\BY}\@pfstepnum{1}{2} ,\,\@pfstepnum{1}{3} ,\,\@pfstepnum{2}{3} ,\,
 CardinalityInNat {\DEF} P}%
\@x{\makebox[0pt][r]{\scriptsize 252\hspace{1em}}\@pfstepnum{1}{q.}\  {\QED}}%
 \@x{\makebox[0pt][r]{\scriptsize 253\hspace{1em}}\@s{8.2}
 {\BY}\@pfstepnum{1}{3} ,\,\@pfstepnum{1}{4}\ }%
\@x{\makebox[0pt][r]{\scriptsize 254\hspace{1em}}}\midbar\@xx{}%
\@pvspace{8.0pt}%
 \@x{\makebox[0pt][r]{\scriptsize 256\hspace{1em}} {\THEOREM} CardinalityUnion
 \.{\defeq}}%
 \@x{\makebox[0pt][r]{\scriptsize 257\hspace{1em}}\@s{55.02} \A\, S ,\, T
 \.{:} IsFiniteSet ( S ) \.{\land} IsFiniteSet ( T ) \.{\implies}}%
 \@x{\makebox[0pt][r]{\scriptsize 258\hspace{1em}}\@s{100.19} \.{\land}
 IsFiniteSet ( S \.{\cup} T )}%
 \@x{\makebox[0pt][r]{\scriptsize 259\hspace{1em}}\@s{100.19} \.{\land}
 IsFiniteSet ( S \.{\cap} T )}%
 \@x{\makebox[0pt][r]{\scriptsize 260\hspace{1em}}\@s{100.19} \.{\land}
 Cardinality ( S \.{\cup} T ) \.{=}}%
 \@x{\makebox[0pt][r]{\scriptsize 261\hspace{1em}}\@s{131.80} Cardinality ( S
 ) \.{+} Cardinality ( T )}%
 \@x{\makebox[0pt][r]{\scriptsize 262\hspace{1em}}\@s{131.80} \.{-}
 Cardinality ( S \.{\cap} T )}%
\@pvspace{8.0pt}%
\@x{\makebox[0pt][r]{\scriptsize 264\hspace{1em}}}\midbar\@xx{}%
\@pvspace{8.0pt}%
 \@x{\makebox[0pt][r]{\scriptsize 266\hspace{1em}} {\THEOREM} PigeonHole
 \.{\defeq}}%
 \@x{\makebox[0pt][r]{\scriptsize 267\hspace{1em}}\@s{63.22} \A\, S ,\, T
 \.{:} \.{\land} IsFiniteSet ( S )}%
 \@x{\makebox[0pt][r]{\scriptsize 268\hspace{1em}}\@s{100.19} \.{\land}
 IsFiniteSet ( T )}%
 \@x{\makebox[0pt][r]{\scriptsize 269\hspace{1em}}\@s{100.19} \.{\land}
 Cardinality ( T ) \.{<} Cardinality ( S )}%
 \@x{\makebox[0pt][r]{\scriptsize 270\hspace{1em}}\@s{100.19} \.{\implies}
 \A\, f \.{\in} [ S \.{\rightarrow} T ] \.{:}}%
 \@x{\makebox[0pt][r]{\scriptsize 271\hspace{1em}}\@s{123.94} \E\, x ,\, y
 \.{\in} S \.{:} x \.{\neq} y \.{\land} f [ x ] \.{=} f [ y ]}%
 \@x{\makebox[0pt][r]{\scriptsize 272\hspace{1em}}\@pfstepnum{1}{}\  {\DEFINE}
 P ( n ) \.{\defeq} \A\, S \.{:} IsFiniteSet ( S ) \.{\land} ( Cardinality (
 S ) \.{=} n ) \.{\implies}}%
 \@x{\makebox[0pt][r]{\scriptsize 273\hspace{1em}}\@s{99.88} \A\, T \.{:}
 \.{\land} IsFiniteSet ( T )}%
 \@x{\makebox[0pt][r]{\scriptsize 274\hspace{1em}}\@s{123.92} \.{\land}
 Cardinality ( T ) \.{<} Cardinality ( S )}%
 \@x{\makebox[0pt][r]{\scriptsize 275\hspace{1em}}\@s{123.92} \.{\implies}
 \A\, f \.{\in} [ S \.{\rightarrow} T ] \.{:}}%
 \@x{\makebox[0pt][r]{\scriptsize 276\hspace{1em}}\@s{147.68} \E\, x ,\, y
 \.{\in} S \.{:} x \.{\neq} y \.{\land} f [ x ] \.{=} f [ y ]}%
\@pvspace{8.0pt}%
 \@x{\makebox[0pt][r]{\scriptsize 278\hspace{1em}}\@pfstepnum{1}{2.}\ 
 {\SUFFICES} \A\, n \.{\in} Nat \.{:} P ( n )}%
 \@x{\makebox[0pt][r]{\scriptsize 279\hspace{1em}}\@s{8.2} {\BY}
 CardinalityInNat}%
 \@x{\makebox[0pt][r]{\scriptsize 280\hspace{1em}}\@pfstepnum{1}{3.}\  P ( 0
 )}%
 \@x{\makebox[0pt][r]{\scriptsize 281\hspace{1em}}\@s{8.2} {\BY}
 CardinalityInNat ,\, SMT}%
 \@x{\makebox[0pt][r]{\scriptsize 282\hspace{1em}}\@pfstepnum{1}{4.}\ 
 {\ASSUME} {\NEW} n \.{\in} Nat ,\, P ( n )}%
 \@x{\makebox[0pt][r]{\scriptsize 283\hspace{1em}}\@s{23.88} {\PROVE}\@s{4.1}
 P ( n \.{+} 1 )}%
 \@x{\makebox[0pt][r]{\scriptsize 284\hspace{1em}}\@s{8.2}\@pfstepnum{2}{}\ 
 {\SUFFICES} {\ASSUME} {\NEW} S ,\, IsFiniteSet ( S ) ,\, Cardinality ( S
 )\@s{4.99} \.{=} n \.{+} 1 ,\,}%
 \@x{\makebox[0pt][r]{\scriptsize 285\hspace{1em}}\@s{106.84} {\NEW} T ,\,
 IsFiniteSet ( T ) ,\, Cardinality ( T ) \.{<} Cardinality ( S ) ,\,}%
 \@x{\makebox[0pt][r]{\scriptsize 286\hspace{1em}}\@s{106.84} {\NEW} f \.{\in}
 [ S \.{\rightarrow} T ]}%
 \@x{\makebox[0pt][r]{\scriptsize 287\hspace{1em}}\@s{76.79} {\PROVE}\@s{4.1}
 \E\, x ,\, y \.{\in} S \.{:} x \.{\neq} y \.{\land} f [ x ] \.{=} f [ y ]}%
\@x{\makebox[0pt][r]{\scriptsize 288\hspace{1em}}\@s{16.4} {\OBVIOUS}}%
 \@x{\makebox[0pt][r]{\scriptsize 289\hspace{1em}}\@s{8.2}\@pfstepnum{2}{2.}\ 
 {\PICK} z \.{:} z \.{\in} S}%
 \@x{\makebox[0pt][r]{\scriptsize 290\hspace{1em}}\@s{16.4}\@pfstepnum{3}{1.}\
 S \.{\neq} \{ \}}%
 \@x{\makebox[0pt][r]{\scriptsize 291\hspace{1em}}\@s{24.59} {\BY}
 CardinalityZero ,\, IsaM (\@w{force} )}%
 \@x{\makebox[0pt][r]{\scriptsize 292\hspace{1em}}\@s{16.4}\@pfstepnum{3}{2.}\
 {\QED}}%
 \@x{\makebox[0pt][r]{\scriptsize 293\hspace{1em}}\@s{24.59}
 {\BY}\@pfstepnum{3}{1}\ }%
 \@x{\makebox[0pt][r]{\scriptsize 294\hspace{1em}}\@s{8.2}\@pfstepnum{2}{3.}
 {\CASE} \E\, w \.{\in} S \.{:} w \.{\neq} z \.{\land} f [ w ] \.{=} f [ z ]}%
 \@x{\makebox[0pt][r]{\scriptsize 295\hspace{1em}}\@s{16.4}
 {\BY}\@pfstepnum{2}{2} ,\,\@pfstepnum{2}{3}\ }%
 \@x{\makebox[0pt][r]{\scriptsize 296\hspace{1em}}\@s{8.2}\@pfstepnum{2}{4.}
 {\CASE} \A\, w\@s{4.1} \.{\in} S \.{:} w \.{\neq} z \.{\implies} f [ w ]
 \.{\neq} f [ z ]}%
 \@x{\makebox[0pt][r]{\scriptsize 297\hspace{1em}}\@s{16.4}\@pfstepnum{3}{1.}\
 {\DEFINE} g \.{\defeq} [ w \.{\in} ( S \.{\,\backslash\,} \{ z \} )
 \.{\mapsto} f [ w ] ]}%
 \@x{\makebox[0pt][r]{\scriptsize 298\hspace{1em}}\@s{16.4}\@pfstepnum{3}{2.}\
 \E\, x ,\, y \.{\in} S \.{\,\backslash\,} \{ z \} \.{:} x \.{\neq} y
 \.{\land} g [ x ] \.{=} g [ y ]}%
 \@x{\makebox[0pt][r]{\scriptsize 299\hspace{1em}}\@s{24.59}\@pfstepnum{4}{1.}
 \.{\land} IsFiniteSet ( S\@s{2.59} \.{\,\backslash\,} \{ z \} )}%
 \@x{\makebox[0pt][r]{\scriptsize 300\hspace{1em}}\@s{45.15} \.{\land}
 Cardinality ( S\@s{1.66} \.{\,\backslash\,} \{ z \} ) \.{=} ( n \.{+} 1 )
 \.{-} 1}%
 \@x{\makebox[0pt][r]{\scriptsize 301\hspace{1em}}\@s{45.15} \.{\land}
 IsFiniteSet ( T\@s{0.92} \.{\,\backslash\,} \{ f [ z ] \} )}%
 \@x{\makebox[0pt][r]{\scriptsize 302\hspace{1em}}\@s{45.15} \.{\land}
 Cardinality ( T \.{\,\backslash\,} \{ f [ z ] \} ) \.{=} Cardinality ( T )
 \.{-} 1}%
\@x{\makebox[0pt][r]{\scriptsize 303\hspace{1em}}\@s{32.8} {\BY}}%
\@y{%
 \ensuremath{\@pfstepnum{2}{1}},
}%
\@xx{\@pfstepnum{2}{2} ,\, CardinalitySetMinus}%
 \@x{\makebox[0pt][r]{\scriptsize
 304\hspace{1em}}\@s{24.59}\@pfstepnum{4}{2.}\  Cardinality ( T
 \.{\,\backslash\,} \{ f [ z ] \} ) \.{<} Cardinality ( S \.{\,\backslash\,}
 \{ z \} )}%
\@x{\makebox[0pt][r]{\scriptsize 305\hspace{1em}}\@s{32.8} {\BY}}%
\@y{%
 \ensuremath{\@pfstepnum{2}{1}},
}%
\@xx{ CardinalityInNat ,\,\@pfstepnum{4}{1} ,\, SMT}%
 \@x{\makebox[0pt][r]{\scriptsize
 306\hspace{1em}}\@s{24.59}\@pfstepnum{4}{3.}\  \A\, ff \.{\in} [ S
 \.{\,\backslash\,} \{ z \} \.{\rightarrow} T \.{\,\backslash\,} \{ f [ z ]
 \} ] \.{:}}%
 \@x{\makebox[0pt][r]{\scriptsize 307\hspace{1em}}\@s{63.91} \E\, x ,\, y
 \.{\in} S \.{\,\backslash\,} \{ z \} \.{:} x \.{\neq} y \.{\land} ff [ x ]
 \.{=} ff [ y ]}%
 \@x{\makebox[0pt][r]{\scriptsize 308\hspace{1em}}\@s{32.8}
 {\BY}\@pfstepnum{1}{4} ,\,\@pfstepnum{4}{1} ,\,\@pfstepnum{4}{2} ,\, IsaM
 (\@w{auto} )}%
 \@x{\makebox[0pt][r]{\scriptsize
 309\hspace{1em}}\@s{24.59}\@pfstepnum{4}{4.}\  g \.{\in} [ S
 \.{\,\backslash\,} \{ z \} \.{\rightarrow} T \.{\,\backslash\,} \{ f [ z ]
 \} ]}%
 \@x{\makebox[0pt][r]{\scriptsize 310\hspace{1em}}\@s{32.8}
 {\BY}\@pfstepnum{2}{4}\ }%
 \@x{\makebox[0pt][r]{\scriptsize 311\hspace{1em}}\@s{24.59}\@pfstepnum{4}{}\ 
 {\HIDE} {\DEF} g}%
 \@x{\makebox[0pt][r]{\scriptsize
 312\hspace{1em}}\@s{24.59}\@pfstepnum{4}{5.}\  {\QED}}%
 \@x{\makebox[0pt][r]{\scriptsize 313\hspace{1em}}\@s{32.8}
 {\BY}\@pfstepnum{4}{4} ,\,\@pfstepnum{4}{3}\ }%
 \@x{\makebox[0pt][r]{\scriptsize 314\hspace{1em}}\@s{16.4}\@pfstepnum{3}{3.}\
 \@s{4.1} {\QED}}%
 \@x{\makebox[0pt][r]{\scriptsize 315\hspace{1em}}\@s{24.59}
 {\BY}\@pfstepnum{3}{2}\ }%
 \@x{\makebox[0pt][r]{\scriptsize 316\hspace{1em}}\@s{8.2}\@pfstepnum{2}{5.}\ 
 {\QED}}%
 \@x{\makebox[0pt][r]{\scriptsize 317\hspace{1em}}\@s{16.4}
 {\BY}\@pfstepnum{2}{3} ,\,\@pfstepnum{2}{4}\ }%
 \@x{\makebox[0pt][r]{\scriptsize 318\hspace{1em}}\@pfstepnum{1}{}\  {\HIDE}
 {\DEF} P}%
\@x{\makebox[0pt][r]{\scriptsize 319\hspace{1em}}\@pfstepnum{1}{5.}\  {\QED}}%
 \@x{\makebox[0pt][r]{\scriptsize 320\hspace{1em}}\@s{8.2}
 {\BY}\@pfstepnum{1}{3} ,\,\@pfstepnum{1}{4} ,\, NatInduction}%
\@x{\makebox[0pt][r]{\scriptsize 321\hspace{1em}}}\midbar\@xx{}%
\@pvspace{8.0pt}%
 \@x{\makebox[0pt][r]{\scriptsize 323\hspace{1em}} {\THEOREM} \A\, S ,\, T ,\,
 f \.{:}\@s{4.1} \.{\land} IsFiniteSet ( S )}%
 \@x{\makebox[0pt][r]{\scriptsize 324\hspace{1em}}\@s{99.19} \.{\land} f
 \.{\in} [ S \.{\rightarrow} T ]}%
 \@x{\makebox[0pt][r]{\scriptsize 325\hspace{1em}}\@s{99.19} \.{\land} \A\, y
 \.{\in} T \.{:} \E\, x \.{\in} S \.{:} y \.{=} f [ x ]}%
 \@x{\makebox[0pt][r]{\scriptsize 326\hspace{1em}}\@s{99.19} \.{\implies}
 \.{\land} IsFiniteSet ( T )}%
 \@x{\makebox[0pt][r]{\scriptsize 327\hspace{1em}}\@s{114.74} \.{\land}
 Cardinality ( T ) \.{\leq} Cardinality ( S )}%
\@x{\makebox[0pt][r]{\scriptsize 328\hspace{1em}} {\PROOF} {\OMITTED}}%
\@pvspace{8.0pt}%
 \@x{\makebox[0pt][r]{\scriptsize 330\hspace{1em}} {\THEOREM} ProductFinite
 \.{\defeq}}%
 \@x{\makebox[0pt][r]{\scriptsize 331\hspace{1em}}\@s{20.5} \A\, S ,\, T \.{:}
 IsFiniteSet ( S ) \.{\land} IsFiniteSet ( T ) \.{\implies} IsFiniteSet ( S
 \.{\times} T )}%
\@x{\makebox[0pt][r]{\scriptsize 332\hspace{1em}} {\PROOF} {\OMITTED}}%
\@pvspace{8.0pt}%
 \@x{\makebox[0pt][r]{\scriptsize 334\hspace{1em}} {\THEOREM} SubsetsFinite
 \.{\defeq} \A\, S \.{:} IsFiniteSet ( S ) \.{\implies} IsFiniteSet (
 {\SUBSET} S )}%
\@x{\makebox[0pt][r]{\scriptsize 335\hspace{1em}} {\PROOF} {\OMITTED}}%
\@x{\makebox[0pt][r]{\scriptsize 336\hspace{1em}}}\bottombar\@xx{}%
\end{document}
